
\documentclass[letterpaper,12pt]{article}
%documentclass[superscriptaddress,preprintnumbers,amsmath,amssymb,aps,11pt]{revtex4}
%\usepackage[]{authblk}
%\usepackage{graphics}
\usepackage[dvipdf]{graphicx}
%\usepackage{subfig}  % For subfloats
\usepackage{color}

\usepackage{epsfig}
\usepackage{wrapfig}
\usepackage{rotating}
\usepackage{caption}
\usepackage{subcaption}

\oddsidemargin = -14mm
\textwidth = 19cm

\def \rarr {\rightarrow}
\def \grinp {\includegraphics}
\def \tw {\textwidth}
\def\dfrac#1#2{\displaystyle{{#1}\over{#2}}}
\def \dstl {\displaystyle}
\definecolor{GREEN}{rgb}{0.,0.8,0}
\definecolor{RED}{rgb}{1,0,0}
\definecolor{ORANGE}{rgb}{1,0.5,0}

\title{Beam trip sudies using Struck scalers}

\begin{document}
\maketitle
 
 One of the features of HPS experiment is that, in nominal production running condition
 the distance between the beam and the physical edge of the 1st layer of Silicon Vertex Tracker (SVT) is
 $0.5\;mm$, and the silicon sensor is far from the beam by $1.5\;mm$. Beam vertical movement more than $1.5\;mm$ 
 might damage the silicon sensor.
 
 One of the potential sources of beam movement is RF trip, when one of RF cavities failes causing
 the electron gained energy at the hall be slightly less from the nominal value. This energy deviation
 also will cause a change of the beam trajectory.
 
 The goal of this work is to study whether there is a significant beam movement especially in a vertical direction
 during beam trips.
 
 
 
\end{document}
