\documentclass[letterpaper,12pt]{article}
%documentclass[superscriptaddress,preprintnumbers,amsmath,amssymb,aps,11pt]{revtex4}
%\usepackage[]{authblk}
%\usepackage{graphics}
\usepackage[dvipdf]{graphicx}
%\usepackage{subfig}  % For subfloats
\usepackage{color}
\usepackage[usenames,dvipsnames]{xcolor}
\usepackage{epsfig}
\usepackage{wrapfig}
\usepackage{rotating}
\usepackage{caption}
%\usepackage{subcaption}
\usepackage{subfig}
\usepackage{authblk}

\oddsidemargin = -14mm
\topmargin = -2.9cm
\textwidth = 19cm
\textheight = 24cm

\def \rarr {\rightarrow}
\def \grinp {\includegraphics}
\def \tw {\textwidth}
\def\dfrac#1#2{\displaystyle{{#1}\over{#2}}}
\def \dstl {\displaystyle}
\definecolor{GREEN}{rgb}{0.,0.8,0}
\definecolor{RED}{rgb}{1,0,0}
\definecolor{ORANGE}{rgb}{1,0.5,0}

\title{Beam motion studies}

\author[1]{H. Egiyan}
\author[2]{R. Paremuzyan}
\author[1] {S. Stepanyan}

\affil[1]{Thomas Jefferson National Accelerator Facility, Newport News, VA 23606}
\affil[2] {University of New Hampshire,  Durham, NH 03824}

\begin{document}
\maketitle
  The goal of this work is to study whether there is a significant beam movement especially 
  during beam trips.

 One of the potential sources of beam movement is RF trip, when one of RF cavities fails causing
 the electron gained energy at the hall be slightly less from the nominal value. This energy deviation
 also will cause a change of the beam trajectory.
  
  If beam movement is happening in a long time interval (orders of seconds), and it is sufficiently large (orders of mm) that can
 hit beam-pipe, then it will cause a high rate on Halo counters.
 These long term movements can be studied through MYA archiving tool, by correlating BPM positions with rates on Halo
 counters.

 Beam motion can happen in a short time intervals (orders of microseconds) also.  In this case
 BMP readout frequency is too low to be able to catch such fast beam movement. Process Variables (PVs) representing rates on Halo counters, 
 that are stored in MYA are also updating slow (mostly updated every 2 sec in HPS run period).
  
 \section{Connections and characteristics of Struck scalers}
 A code is developed for a Struck Scaler (SIS 3800), that allows to write a buffer of each channel to
 a root file. Each buffer has 60000 elements, and the minimum dwell time that was achieved is $15\; \mu s$.
 When the buffer is filled it takes about $1.1\; s$ to write them into a file. During that time Scaler doesn't fill the buffer.
 Along with the $60\;K$ long buffer it also writes the time-stamp with a nano-second precision for each buffer, that corresponds
 to the local time of the 1-st element of the buffer.
  %%%%%%%%%%%%%%%%%%%%%%%%%%%%%%%%%%%%%%%%%%%%%% F I G U R E %%%%%%%%%%%%%%%%%%%%%%%%%%%%%%%%%%%%%%%%%%
 \begin{figure}[!htb]
  \centering
  \grinp[width=0.95\tw]{img/Struck_Readout.pdf}
  \caption{Readout of Struck scaler}
  \label{fig:Struck_Readout}
 \end{figure}
%%%%%%%%%%%%%%%%%%%%%%%%%%%%%%%%%%%%%%%%%%%%%% F I G U R E %%%%%%%%%%%%%%%%%%%%%%%%%%%%%%%%%%%%%%%%%%
 Schematically it is represented in Fig.\ref{fig:Struck_Readout}.
 
 In the output Root File, 5 channels of Struck SIS 3800 are written. The correspondence of scaler channels and signals are shown
 in Table \ref{tb:scaler_signal_correspondance}.
%%%%%%%%%%%%%%%%%%%%%%%%%%%%%%%%%%%%%%%%%%%%%% T A B L E %%%%%%%%%%%%%%%%%%%%%%%%%%%%%%%%%%%%%%%%%%%% 
 \begin{table}[!htb]
 \centering
  \begin{tabular}{|c|c|}
   \hline
   Channel 0 & $25\; MHz$  clock + ECal crystal \\ \hline
   Channel 1 & "HPS-SC" + ECal crystal \\ \hline
   Channel 2 & "Tagger-top" + "Upstr.-Right" + ECal crystal \\ \hline
   Channel 3 & "HPS-R" + ECal crystal \\ \hline
   Channel 4 &  ECal crystal \\ \hline
  \end{tabular}
  \caption{ Correspondence between scaler channels and signal sources.}
  \label{tb:scaler_signal_correspondance}
 \end{table}
%%%%%%%%%%%%%%%%%%%%%%%%%%%%%%%%%%%%%%%%%%%%%% T A B L E %%%%%%%%%%%%%%%%%%%%%%%%%%%%%%%%%%%%%%%%%%%% 
%%%%%%%%%%%%%%%%%%%%%%%%%%%%%%%%%%%%%%%%%%%%%% F I G U R E %%%%%%%%%%%%%%%%%%%%%%%%%%%%%%%%%%%%%%%%%%
\begin{figure}[!htb]
\centering
\grinp[width=0.9\tw]{img/Struck_Connections.pdf}
\caption{Schematic picture of connections: shown locations of all counters that are connected
to scaler.}
\label{fig:schematics_Connections}
\end{figure}
%%%%%%%%%%%%%%%%%%%%%%%%%%%%%%%%%%%%%%%%%%%%%% F I G U R E %%%%%%%%%%%%%%%%%%%%%%%%%%%%%%%%%%%%%%%%%%
 Schematic picture representing locations of counters is shown in Fig.\ref{fig:schematics_Connections}.
  %%%%%%%%%%%%%%%%%%%%%%%%%%%%%%%%%%%%%%%%%%%%% F I G U R E %%%%%%%%%%%%%%%%%%%%%%%%%%%%%%%%%%%%%%%%%%
 \begin{figure}[!htb]
  \centering
  \grinp[width=0.75\tw]{img/25MHz_Clock.png}
  \caption{3 consequent buffers of $25 \; MHz$ clock. in a $15\;\mu s$ interval it correspond to 375 counts. The gap between readings
  is the dead time, when scaler writes out the data.}
  \label{fig:Clock_Channel}
 \end{figure}
%%%%%%%%%%%%%%%%%%%%%%%%%%%%%%%%%%%%%%%%%%%%% F I G U R E %%%%%%%%%%%%%%%%%%%%%%%%%%%%%%%%%%%%%%%%%%
  The 1-st channel is connected to clock intentionally. That will allow to check whether the scaler reads and writes
 data properly.  Particularly in Fig.\ref{fig:Clock_Channel} shown the clock (channel-0) distribution as a function of time.
 $25\; MHz$ in a $15\;\mu s$ time interval corresponds to 375 counts.
 The gap between readings is the dead time, when scaler writes out the data into the file.
 We noticed also the time-stamps are not correct, since the actual dead time is about $1.1\;s$, but
 in the figure one can see it is about $\approx 150\; ms$.
\section{Testing the system in the upstream} 
 1-st check of the system in the presence of beam was on March-14 and March-15. At that time CEBAF machine was
 in a state that downstream BPMs were not yet calibrated, and the beam direction was to the tagger dump.
 At that moment only Upstream counters were counting high and consequently, the channel 2, which is the only channel that
 connected to upstream counters, had noticeable data in it. Also orbit locks were not engaged yet, and that caused
 strong beam excursions.
  Since the dwell time is quite small ($15\;\mu s$), one need to have a sufficient rates on counters to
 be able to observe rate evolution upon the time. 
  For this reason the vertical thick wire ($1\;mm$ in diameter) of harp\_tagger (wire is parallel to $X$ Axis) was moved to the 
  beam close enough to create about $3 \;MHz$ rates on together "Upstream-L" + "Tagger-Top".
   %%%%%%%%%%%%%%%%%%%%%%%%%%%%%%%%%%%%%%%%%%%%% F I G U R E %%%%%%%%%%%%%%%%%%%%%%%%%%%%%%%%%%%%%%%%%%
 \begin{figure}[!htb]
\centering
 \grinp[width=0.95\tw]{Upstream_Rate_Pos_Correlation.pdf}
 \caption{A figure showing periodic variation of rates on Upstream-Right halo counter, when harp thick horizontal
 wire is touching the beam from the top. It is directly related to the
 Beam Y position measured at 2C24A.}
 \label{fig:upstr_rate_Variation.}
\end{figure}
 %%%%%%%%%%%%%%%%%%%%%%%%%%%%%%%%%%%%%%%%%%%%% F I G U R E %%%%%%%%%%%%%%%%%%%%%%%%%%%%%%%%%%%%%%%%%%
 In Fig.\ref{fig:upstr_rate_Variation.} a small range of a stripchart from MYA archiver is shown, where
  2C24 $Y$ position is indicated by green, Red represents rates on Upstream-R counter, and the brown is the beam current
  at 2C24A.
 As one can see beam current in that region is quite stable, instead beam position is oscillating with a period of about
 $30\;s$, and it is in correlation with rates on Upstream Halo counters. When beam moves top, and consequently touches
 the wire, rates quickly rise up to few $MHz$.
  %%%%%%%%%%%%%%%%%%%%%%%%%%%%%%%%%%%%%%%%%%%%% F I G U R E %%%%%%%%%%%%%%%%%%%%%%%%%%%%%%%%%%%%%%%%%%
 \begin{figure}
  \centering
  \grinp[width=0.95\tw]{img/Upstream_rates1.png}
  \caption{An example of one buffer with 60K readings from scaller channel 2.}
  \label{fig:upstr_counts1}
 \end{figure}
 %%%%%%%%%%%%%%%%%%%%%%%%%%%%%%%%%%%%%%%%%%%%% F I G U R E %%%%%%%%%%%%%%%%%%%%%%%%%%%%%%%%%%%%%%%%%% 
 In Fig.\ref{fig:upstr_counts1} shown an example of one buffer with 60K readings from scaler channel 2.
 As one can notice from the figure, rates on scalers oscillate a lot. Rates vary from 0 up to 140. 
 140 counts correspond to about $9\;MHz$. Taking into account that $9\;MHz$ is the sum of two counters
"Upstream-R" and "Tagger-top". These two counters count roughly equal. So the maximum rate would correspond
to about $4.5\;MHz$ to each counter. 
The rate of oscillations isn't appear to be constant, but it mostly within $60\;Hz$ to $100\;Hz$.
So taking into account above mentioned one can conclude that, without orbit locks at 2C24, there
exist two beam motion components. One is fast (within $60\;Hz$ to $100\;Hz$), another
is slow (with a 30 sec period). According to BPMs beam motion at 2C24 side is about $500\; \mu m$,
however this doesn't mean that the beam motion will be the same order in the 2H02 side (close to HPS detector),
since settings of quadrapole magnets are such, that in Hall B beam has the largest spatial size at 2C24,
whereas it is around $40\; \mu m$ in the 2H02A side.
\section{Beam motion studies at HPS side}
Downstream BPMs were calibrated and orbit locks were engaged on Apr 19 only,
after the restoration of the machine from the power outage, that caused failure of CHL2.
Next set of beam movement studies with Struck scalers began on Apr 27 owl.
In a similar way as in upstream part, at HPS side also to determine
the beam position, the 2H02A harp
vertical thick wire ($1\;mm$ in diameter) was passed through the beam.
 %%%%%%%%%%%%%%%%%%%%%%%%%%%%%%%%%%%%%%%%%%%%% F I G U R E %%%%%%%%%%%%%%%%%%%%%%%%%%%%%%%%%%%%%%%%%%
\begin{figure}[!htb]
 \centering
 \grinp[width=0.95\tw]{img/FCup_harp_halo_correlation1.png}
 \caption{Scan of the beam vertical profile with a harp $1\;mm$ wire. By blue shown harp motor position,
by green shown rates on HPS\_SC counter, by red shown rates on Downstream\_bottom counter
 and the dashed orange line represents FCup current.}
\label{fig:2H02A_thick_wire_scan}
\end{figure}
 %%%%%%%%%%%%%%%%%%%%%%%%%%%%%%%%%%%%%%%%%%%%% F I G U R E %%%%%%%%%%%%%%%%%%%%%%%%%%%%%%%%%%%%%%%%%%
In Fig.\ref{fig:2H02A_thick_wire_scan} shown time dependence of the Faraday Cup current (dashed orange), 
rates on HPS\_SC halo counter (green), rates on Downstream\_bottom counter (red) and the harp\_2H02A 
motor position (blue), during the scan.
 One can notice an apparent correlation between these variables. When the harp wire is far from the beam
 FCup current is stable (around $30\; nA$), both halo counters count quite small. As soon wire gets closer
 to the beam FCup current start to decrease, both halo counters start to increase. Reaching $\approx 600\;KHz$ HPS\_SC
 saturates causing counts to decrease. We don't see saturation on Downstream\_bottom, because it is upstream w.r.t. the 
 harp\_2H02A, and consequently will have much lower rates (The peak value is $\approx 45\;Hz$).
 The fact that FCup current decrease we think is because the $1\; mm$ wire plays a role of a small "beam blocker", and after the 
 wire about $30\%$ of beam electrons have scattered at enough large angles, or have such low energy that chicane bends
 them from the nominal trajectory causing them to miss FCup. For not being too busy plot the beam current measured at 2C24
 is not plotted, but it is quite stable, confirming additionally that the FCup current change is not because of the actual
 current decrease, but is because of the harp wire and the beam interaction.
 %%%%%%%%%%%%%%%%%%%%%%%%%%%%%%%%%%%%%%%%%%%%% F I G U R E %%%%%%%%%%%%%%%%%%%%%%%%%%%%%%%%%%%%%%%%%%
 \begin{figure}[!htb]
  \centering
  \grinp[width=0.95\tw]{img/FCup_Halo_corr_With_Fix_Harpwire.pdf}
  \caption{Time dependence of HPS\_SC (green), FCup scaler (dashed orange), and beam X (dashed blue) and Y (Pink) in a 3 min
  time interval.}
  \label{fig:FCup_Halo_pos_time_dep1}
 \end{figure}
 %%%%%%%%%%%%%%%%%%%%%%%%%%%%%%%%%%%%%%%%%%%%% F I G U R E %%%%%%%%%%%%%%%%%%%%%%%%%%%%%%%%%%%%%%%%%%
 After the scan, beam blocker was moved in front of the FCup, then by our request MCC delivered us $100\;nA$ beam, 
 and harp vertical $1\;mm$ wire was positioned right at the place (around $10.55\;cm$) where rates on HPS\_SC start to increase.
 Now when BPMs are calibrated and orbit locks are engaged, no periodic $30\;s$ rate oscillation was observed, 
 as was observed in upstream part without engaged orbit locks. Actually 
 during that shift few times the harp wire position was adjusted by $\approx 50\;\mu m$ to make sure that rates on HPS\_SC
 counter are about $2-3\; MHz$, however this doesn't have a periodic character. 
 In Fig.\ref{fig:FCup_Halo_pos_time_dep1} shown time dependence of HPS\_SC (green), 
 FCup scaler (dashed orange), beam X (dashed blue) and Y (Pink) in a 3 min time interval. Note here instead of plotting FCup
 measured current value, the FCup scaler value is plotted. This is because with beam blocker
 the FCup current is small (around $0.9\; nA$), and small current changes are below MYA dead band (it is $0.01$ for this channel).
 The FCup scaler value is almost linearly proportional to the FCup current, and therefore for correlation purposes 
 it can be used instead of FCup current.
 One can note that here also (when the beam blocker is under the beam), 
 there is a correlation between FCup current and rates on Halo counters, however the correlation
 now is opposite, i.e. increase of count rates correspond to the increase of FCup current.
 The explanation for that we think is the following: when fraction of the beam (or entire beam) is interacting
 with the harp wire, then after the wire the beam is not mono-energetic and parallel anymore, in addition to that
 the chicane makes that dispersion even bigger, causing some part of the beam electrons escape the blocker and enter into FCUP.
 %%%%%%%%%%%%%%%%%%%%%%%%%%%%%%%%%%%%%%%%%%%%% F I G U R E %%%%%%%%%%%%%%%%%%%%%%%%%%%%%%%%%%%%%%%%%%
 \begin{figure}[!htb]
  \centering
  \grinp[width=0.95\tw]{img/FCup_Blocker_diagram.pdf}
  \caption{A qualitative description of how the FCup current increase with the presence of the beam blocker,
  when the beam interacts with a harp wire. Left represents the case when beam is not interacting with any matarial
  before the beam blocker. Righ figure: beam has interacted with a harp wire.}
  \label{fig:FCup_blocker_diagram}
 \end{figure}
 %%%%%%%%%%%%%%%%%%%%%%%%%%%%%%%%%%%%%%%%%%%%% F I G U R E %%%%%%%%%%%%%%%%%%%%%%%%%%%%%%%%%%%%%%%%%%
It is qualitatively illustrated in Fig.\ref{fig:FCup_blocker_diagram}.
Left drawing represents intact beam case, where almost all the beam hit the blocker, and in the
right drawing beam has interacted with the harp before, and some of electrons will now miss the beam blocker and enter
FCUP.
 %%%%%%%%%%%%%%%%%%%%%%%%%%%%%%%%%%%%%%%%%%%%% F I G U R E %%%%%%%%%%%%%%%%%%%%%%%%%%%%%%%%%%%%%%%%%%
 \begin{figure}[!htb]
  \centering
  \grinp[width=0.95\tw]{img/BPM_Halo_2H02_1.pdf}
  \caption{Stripchart representing Rates on HPS\_SC (green) and 2H02 BPM\_Y (red) when 2H02A $1\;mm$ vertical wire is touching the beam from the top}
  \label{fig:2H02_rates_pos_correl1}
 \end{figure}
 %%%%%%%%%%%%%%%%%%%%%%%%%%%%%%%%%%%%%%%%%%%%% F I G U R E %%%%%%%%%%%%%%%%%%%%%%%%%%%%%%%%%%%%%%%%%%
In Fig.\ref{fig:FCup_Halo_pos_time_dep1} it is hard to see a correlation between beam
vertical position and the rates on HPS\_SC mainly because the scale of vertical position is relatively large, and
noise of BPM readings is large w.r.t. average position change.
In Fig.\ref{fig:2H02_rates_pos_correl1} shown beam vertical position at 2H02 (red) and rates on HPS\_SC (green) with 
appropriate scale and larger time interval. Now the correlation is more visible, also one can see that the beam 
movement in all that time interval is less than $50\;\mu m$.
 %%%%%%%%%%%%%%%%%%%%%%%%%%%%%%%%%%%%%%%%%%%%% F I G U R E %%%%%%%%%%%%%%%%%%%%%%%%%%%%%%%%%%%%%%%%%%
% \begin{figure}[!htb]
%  \centering
%  \grinp[width=0.95\tw]{img/BPM_pos_width_curr_Correl.pdf}
%  \caption{Stripchart representing beam current (blue), beam position at 2H02A (red) and rates on
%  HPS\_SC (green) }
%  \label{fig:Width_curr_dep}
% \end{figure}
 %%%%%%%%%%%%%%%%%%%%%%%%%%%%%%%%%%%%%%%%%%%%% F I G U R E %%%%%%%%%%%%%%%%%%%%%%%%%%%%%%%%%%%%%%%%%%
 The BPM noise level is inversely proportional to the beam current. In the future runnings the nominal 
 production current should be $\approx 200\;nA$, consequently
 the spatial resolution of BPMs will be better, that will allow to estimate beam motions more preciesly.
%%%%%%%%%%%%%%%%%%%%%%%%%%%%%%%%%%%%%%%%%%%%% F I G U R E %%%%%%%%%%%%%%%%%%%%%%%%%%%%%%%%%%%%%%%%%%
\begin{figure}[!htb]
\centering
 \subfloat[]{\label{fig:BPM_Jump_ex1}\grinp[width=0.95\tw]{img/BPM_Jump.pdf}} \\
 \subfloat[]{\label{fig:BPM_Jump_ex2}\grinp[width=0.95\tw]{img/BPM_Jump2.pdf}} \\
 \caption{Strip chart showing a significant beam movement for a few seconds. a) rates on HPS\_SC
 (green), 2H02.Y (pink) and rates on Downstream-bot (blue).
 b) Beam X and Y positions at 2C24 (green and yellow respectively), and at 2H02 (blue and pink respectively).}
 \label{fig_BPM_Jump1}
\end{figure}
%%%%%%%%%%%%%%%%%%%%%%%%%%%%%%%%%%%%%%%%%%%%% F I G U R E %%%%%%%%%%%%%%%%%%%%%%%%%%%%%%%%%%%%%%%%%%
 There were observed few instances also, when there was a beam movement of the order
 of few hundreds of microns. One example is depicted in Fig.\ref{fig_BPM_Jump1}, where one can
 notice correlation between BPM position readings and rates on halo counters, i.e. 
 HPS\_SC is saturated while downstream shows a clear peak during the beam movement.
 While at HPS side beam movement is $\approx 150\; \mu m$ (vertical) and $\approx 500\; \mu m$ (horizontal),
 in upstream the beam movement is more than several $mm$s. The fact that the beam has moved in upstream also
 is an indication that the movement is not caused by the malfunction of Hall-B orbit locks, but
 is coming from the accelerator, and Hall-B orbit locks make it quite smaller at HPS side.
 %%%%%%%%%%%%%%%%%%%%%%%%%%%%%%%%%%%%%%%%%%%%% F I G U R E %%%%%%%%%%%%%%%%%%%%%%%%%%%%%%%%%%%%%%%%%%
 \begin{figure}[!htb]
  \centering
  \grinp[width=0.95\tw]{img/Struck_Oscilations_Zoom.png}
  \caption{Counts as a function of time for Struck scaler channel 1 (HPS\_SC) in $100\;ms$ time
  interval. Each point represents counts in a $15\;\mu s$ time interval.}
  \label{fig:Struck_Oscil_Zoom1}
 \end{figure}
 %%%%%%%%%%%%%%%%%%%%%%%%%%%%%%%%%%%%%%%%%%%%% F I G U R E %%%%%%%%%%%%%%%%%%%%%%%%%%%%%%%%%%%%%%%%%%
 In Fig.\ref{fig:Struck_Oscil_Zoom1} shown counts as a function of time for Struck scaler
 channel 1 (HPS\_SC) in $100\;ms$ time interval. We see there is a clear oscialations of rates.
 Oscilations are not occuring by a constant period, but it is mainly between $50$ to $100\;Hz$.
 From MYA we learnt that at 2H02 in less than 50 $\mu\;m$ beam movement, rates on HPS\_SC increase
 $\approx 10$ times. In Fig.\ref{fig:Struck_Oscil_Zoom1} oscialations are also the same order
 about $10-20$ times between min and max values, so beam fast motions also should be within $50\; \mu m$.
  %%%%%%%%%%%%%%%%%%%%%%%%%%%%%%%%%%%%%%%%%%%%% F I G U R E %%%%%%%%%%%%%%%%%%%%%%%%%%%%%%%%%%%%%%%%%%
 \begin{figure}[!htb]
  \centering
  \grinp[width=0.95\tw]{img/Struck_trip2.png}
  \caption{counts on HPS\_SC from Struck scaler as a function of time. No increase of rates is observed
  during the trip.}
  \label{fig:Trip_quiet1}
 \end{figure}
 %%%%%%%%%%%%%%%%%%%%%%%%%%%%%%%%%%%%%%%%%%%%% F I G U R E %%%%%%%%%%%%%%%%%%%%%%%%%%%%%%%%%%%%%%%%%%
 Looking into Struck data we also paied a special attention on beam trips, to see whether there is
 an increase of rates just before the trip. On owl shift of Jun 27 wei didn't obsere any increase of
 rates just before the trip. One example is shown in Fig.\ref{fig:Trip_quiet1}, where one can see that
 rates just go down to 0 within few $\mu s$. 
 
 \subsection{Beam motions using the collimator}
 Beam motions were studied with HPS collimator also. The idea was to find the ceter,
 of the collimator $3\;mm$ gap w.r.t. the beam, then 
 place the collimator bottom and top edges close to the beam by about $200\; \mu m$,
 and the take some data with Struck scalers, to see whether there is an increase of rates
 on beam trips or in general.
  %%%%%%%%%%%%%%%%%%%%%%%%%%%%%%%%%%%%%%%%%%%%% F I G U R E %%%%%%%%%%%%%%%%%%%%%%%%%%%%%%%%%%%%%%%%%%
  \begin{figure}[!htb]
   \centering
   \grinp[width=0.95\tw]{img/Colim_SCan.pdf}
   \caption{collimator scan with the beam, to determine the center of the collimator $3\;mm$
   gap w.r.t. the beam. In the figure shown motor position (blue), counts on HPS\_SC (gold),
   and counts on HPS\_t (brown).}
   \label{fig:Colim_Scan1}
  \end{figure}
  %%%%%%%%%%%%%%%%%%%%%%%%%%%%%%%%%%%%%%%%%%%%% F I G U R E %%%%%%%%%%%%%%%%%%%%%%%%%%%%%%%%%%%%%%%%%%
 In Fig.\ref{fig:Colim_Scan1} shown the correlation between colimator motor position (blue) and 
 rates on HPS\_SC (gold) and HPS\_t (brown). One can notice as cloimator top edge gets closer
 to the beam, rates on halo counters gradually increase, and on the last step (motor is a $5.096^{"}$), rates increase
 more than 10 times, that already means that the top edge of the colimator is interacting with
 some fraction of the beam. In a similar way in the figure shown the determination of the motor
 position where the bottom edge of the colimator starts to interact with the beam, which is $4.960^{"}$. 
 The clomiator  center is defined as 
 \begin{equation}
  c = \frac{(5.096 + 4.960)}{2} = 5.028
 \end{equation}
One can calculate colimator gap width as $d = (5.096 - 4.960)\cdot 25.4 = 3.454\;mm$.
During Struck studies we assumed $3\;mm$ gap and moved colimtor such that bottom and top 
edges of the colimator be $\approx 200\;\mu m$ away from the beam, but since now it turns out that
colimator gap is about $450 \mu m$ larger, the actual distance of the edges of the collimator from
would be $200 + 450/2. = 425\;\mu m$. With each collimator position about $30$ minutes of Struck
data was taken. During the whole 1 hour running. 6 beam trips were observed. Only in one case
we saw increase of counts.
%%%%%%%%%%%%%%%%%%%%%%%%%%%%%%%%%%%%%%%%%%%%% F I G U R E %%%%%%%%%%%%%%%%%%%%%%%%%%%%%%%%%%%%%%%%%%
\begin{figure}[!htb]
 \centering
 \subfloat[]{\label{fig_colim_HPS_trip}\grinp[width=0.95\tw]{img/Colim_trip_HPS_SC.png}} \\
 \subfloat[]{\label{fig_colim_upstream_trip}\grinp[width=0.95\tw]{img/Colim_trip_upstream.png}}
 \caption{Couns on Upstream (top) and HPS\_SC (bottom) counters during the beam trip.}
 \label{fig:Colim_trip1}
\end{figure}
%%%%%%%%%%%%%%%%%%%%%%%%%%%%%%%%%%%%%%%%%%%%% F I G U R E %%%%%%%%%%%%%%%%%%%%%%%%%%%%%%%%%%%%%%%%%%
That case is represented in Fig.\ref{fig:Colim_trip1}. The graph on top represents HPS\_SC counter,
and the graph on bottom represents "Upstream\_R + tagger\_t" (Reminder 15 counts is equivalent to $1\;MHz$).
Since the peak present not only on HPS\_SC but also on upstream counters, then this means that
the high rate is not only because of the interaction of the beam and the collimator but 
the beam has interacted with some material before reaching the colimator. 

During that running we also observed similar peak on HPS\_SC, but in that case it wasn't related to the beam trip.
Rates just peaked for a short period of time then decreased.
%%%%%%%%%%%%%%%%%%%%%%%%%%%%%%%%%%%%%%%%%%%%% F I G U R E %%%%%%%%%%%%%%%%%%%%%%%%%%%%%%%%%%%%%%%%%%
\begin{figure}
 \centering
 \grinp[width=0.95\tw]{img/Colim_Beam_move.png}
 \caption{Counts on }
\end{figure}
%%%%%%%%%%%%%%%%%%%%%%%%%%%%%%%%%%%%%%%%%%%%% F I G U R E %%%%%%%%%%%%%%%%%%%%%%%%%%%%%%%%%%%%%%%%%%
 There was no related peak obsereved at upstream part. That probably means that the peak is a result of the beam 
 and the collimator interaction. In all 2 cases the characteristic time of the peak is less than $150\;\mu s$.
 
\subsection{Studies with the target "IN"}
We took some parasitic data too with struck scaler, i.e. during production running when target was under the beam.
Of course here there will be a constant background present, that is coming from the target, but if the beam
will hit some thick/dense material then one can expect much higher rates on halo counters.
%%%%%%%%%%%%%%%%%%%%%%%%%%%%%%%%%%%%%%%%%%%%% F I G U R E %%%%%%%%%%%%%%%%%%%%%%%%%%%%%%%%%%%%%%%%%%
\begin{figure}[!htb]
 \centering
 \subfloat[]{\label{fig:trip_quiet_target}\grinp[width=0.45\tw]{img/quiet_Target.png}}
 \subfloat[]{\label{fig:trip_quiet_semitarget}\grinp[width=0.45\tw]{img/Semiquiet_with_target.png}} \\
 \subfloat[]{\label{fig:trip_nonquiet_target_short}\grinp[width=0.45\tw]{img/Non_Queit_Target_quick.png}}
 \subfloat[]{\label{fig:trip_nonquiet_target_long}\grinp[width=0.45\tw]{img/Non_queit_Target.png}}
 \caption{Different scenarios of beam trips, when target was "IN". a) Trip is quiet, b) rates slightly increased
 before the trip. c) in $\approx 100\;\mu s$ rates increased by $\approx 10$ times before the trip,
 d) rates increased by $\approx 10$ times before the trip, then after $\approx 600 \;\mu s$ beam is tripped.}
 \label{fig:trips_with_target1}
\end{figure}
%%%%%%%%%%%%%%%%%%%%%%%%%%%%%%%%%%%%%%%%%%%%% F I G U R E %%%%%%%%%%%%%%%%%%%%%%%%%%%%%%%%%%%%%%%%%%
In Fig.\ref{fig:trips_with_target1} shown different scenarios of beam trips with the presence of the
target. In a) beam trip is quiet that would mean that beam didn't move enough to hit anything before
being completely off. in b) counts were increased twice than normal by about $150\; ns$ then went to 0.
This is likely that the beam-tail has interacted slightly with some material, otherwise as wee saw before
rates would increase more than 10 times. In c) we see more than 10 times increase of rates again in a short
($\approx 150\;ns$) time interval. This probably means that the beam had strong interaction
 with some material before being completely off. The scenario on d) is quite rear in sense that
 the peak time is about $600 \; \mu s$ whereas it is usually within $100-200\;\mu s$. 
 During almost all trips the trip type also was recorded, and we didn't notice a correlation between the trip
 type and the behavior of halo counters, e.g. for the RF trip sometimes beam goes off quietly without making
 high rates on hao counters, and sometimes it is opposite rates on Halo counters jump by an order of magnitude
 before the trip. Same thing is with the BLM trips or other types of trips.
 We should stress in addition that increase of rates doesn't necessarily mean that the beam has interacted
 with the HPS collimator. We already saw from MYA and Struck data several instances where the beam has moved and hit into some
 material in the way upstream.
 
 \section{Testing the FSD system}
 It is clear from the above discussed that even large beam motions are not avoidable. 
 To keep SVT from being exposed by the beam, signals from all HPS halo counters (HPS\_L(R, SC, t)) summed up
 and sent to CEBAF Fast Shut Down (FSD) system. The way how FSD works is, during some time window called integration time,
 it counts number of pulses, and if it is higher than a certain threshold, it sends a signal to the injector to
 shut down the beam. The integration time and the threshold level can be set by the user. 
 Of course as smaller will be the integration time as better, but on the other hand one need
 to have sufficient counts, that statistical fluctuations will not cause trips too frequently.
 Initially we put integration time as $10\; ms$. 
 To test the system we ran the 2H02A harp wire to the beam to increase rates on Halo counters,
 then observed evolution of rates with Struck scaler.
 %%%%%%%%%%%%%%%%%%%%%%%%%%%%%%%%%%%%%%%%%%%%% F I G U R E %%%%%%%%%%%%%%%%%%%%%%%%%%%%%%%%%%%%%%%%%%
 \begin{figure}[!htb]
  \centering
  \grinp[width=0.95\tw]{img/FSD_Masked_Scan.png}
  \caption{Rates on HPS\_SC as a function of time when FSD was masked.}
  \label{fig:Rates_FSD_Masked}
 \end{figure}
 %%%%%%%%%%%%%%%%%%%%%%%%%%%%%%%%%%%%%%%%%%%%% F I G U R E %%%%%%%%%%%%%%%%%%%%%%%%%%%%%%%%%%%%%%%%%%
 To determine what value to put for the threshold we run 1 harp wire to cross the beam, while FSD was masked.
 In Fig.\ref{fig:Rates_FSD_Masked} show evolution of rates when FSD was masked and harp wire crossed the beam.
 Here since the integration time is orders of milliseconds,  we also increased the dwell time of Struck 
 to be $300\;\mu s$, which allowed to have larger number of counts in each time bin. In the figure the constant background 
 around 70 counts at the beginning and at the end is coming from the target, while starting at $t-t_{0}\approx 80 \;s$ 
 beam is hitting the wire and consequently creating high rates.
 The threshold that we put is $3000$ counts. In $10\;ms$ it will correspond to $300\; KHz$, which translates
 into $90$ counts in $300\;\mu s$ time interval.
 %%%%%%%%%%%%%%%%%%%%%%%%%%%%%%%%%%%%%%%%%%%%% F I G U R E %%%%%%%%%%%%%%%%%%%%%%%%%%%%%%%%%%%%%%%%%%
 \begin{figure}[!htb]
  \centering
  \subfloat[]{\label{fig:FSD_10ms}\grinp[width=0.95\tw]{img/FSD_10ms_30nA.pdf}}\\
  \subfloat[]{\label{fig:FSD_5ms}\grinp[width=0.95\tw]{img/FSD_Test.png}}
  \caption{HPS\_SC rates as a function of time. In top FSD integration time is $10\; ms$ and
  the the beam current is $30\;nA$. In the bottom graph FSD integration time is $5\; ms$ and
  the the beam current is $50\;nA$.}
  \label{fig:Rates_FSD_Test}
 \end{figure}
 %%%%%%%%%%%%%%%%%%%%%%%%%%%%%%%%%%%%%%%%%%%%% F I G U R E %%%%%%%%%%%%%%%%%%%%%%%%%%%%%%%%%%%%%%%%%%
 In Fig.\ref{fig:Rates_FSD_Test} represented two trips with two integration times $10\; ms$ (top) and
 $5\;ms$ (bottom). Red line is the translated threshold that would be if integration time is
 equal to the dwell time of the Struck scaler ($300\;\mu s$). One can see that the time
 between threshold crossing and beam shutdown is consistent with the integration time.
 We repeated this test several times with both $5\;ms$ and $10\;ms$ integration times, and 
 and observed consistent results. 
 
  \section{To summarize}
  \begin{itemize}
  
 \item collimator scan with the beam suggests that collimator $3\;mm$ gap is about $3.45\;mm$. It will
 more clear when collimator will be removed and gap size measured precisely.

 \item Two types of beam motions were observed, fast (roughly $100\; Hz$) and slow that BPMs can detect.
  With engaged orbit locks the vertical beam motion is mainly within $50\;\mu m$.
  
 \item Orbit locks keep the beam quite stable, however sometimes it happens during the trip, 
  or even without trip that the beam moves by more than $100\;\mu m$ at HPS side.

 \item The exact reason for the beam to move or not to move before the trip is not clear.
 Analyzing different trips we didn't find a correlation between the trip type and 
 the behavior of halo counters.

 \item The FSD system was tested, and it seems to work properly.

 \end{itemize}
 \end{document}
 