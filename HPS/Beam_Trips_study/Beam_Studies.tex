\documentclass[letterpaper,12pt]{article}
%documentclass[superscriptaddress,preprintnumbers,amsmath,amssymb,aps,11pt]{revtex4}
%\usepackage[]{authblk}
%\usepackage{graphics}
\usepackage[dvipdf]{graphicx}
%\usepackage{subfig}  % For subfloats
\usepackage{color}
\usepackage[usenames,dvipsnames]{xcolor}
\usepackage{epsfig}
\usepackage{wrapfig}
\usepackage{rotating}
\usepackage{caption}
%\usepackage{subcaption}
\usepackage{subfig}
\usepackage{authblk}

\oddsidemargin = -14mm
\topmargin = -2.9cm
\textwidth = 19cm
\textheight = 24cm

\def \rarr {\rightarrow}
\def \grinp {\includegraphics}
\def \tw {\textwidth}
\def\dfrac#1#2{\displaystyle{{#1}\over{#2}}}
\def \dstl {\displaystyle}
\definecolor{GREEN}{rgb}{0.,0.8,0}
\definecolor{RED}{rgb}{1,0,0}
\definecolor{ORANGE}{rgb}{1,0.5,0}

\title{Beam motion studies}

\author[1]{H. Egiyan}
\author[2]{R. Paremuzyan}
\author[1]{S. Stepanyan}

\affil[1]{Thomas Jefferson National Accelerator Facility, Newport News, VA 23606}
\affil[2] {University of New Hampshire,  Durham, NH 03824}

\begin{document}
\maketitle

The goal of this work is to study whether there is any significant beam motion at the microsecond/millisecond level.
Such fast motion can arise, for example, from a RF trip, when one of RF cavities fails, causing a change of the electron energy. 
This energy change will change the beam trajectory to the hall and if dispersion allows will change
the beam position at the detector. Beam trajectory and the position at the detector can change also 
if some of magnetic elements will fail, however this will be much slower motion. 
  
If there is a slow beam position drifts, order of a second or so, the feedback system (known as orbit locks) based
on Beam Position Monitors (BPMs) and a set of corrector magnets should be able to handle it, 
keeping the beam steady at the detector within the reasonable limits
(during HPS spring run this system kept beam position steady within $50~\mu$m). 
In case this system fails or the beam motion is too fast for it, 
there are other safeguards one of which is so called machine Fast ShutDown (FSD) system 
that takes as an input rates of halo counters and will initiate machine shutdown 
if these rates will exceed given threshold within set integration time (order of milliseconds). 
For example if beam motion is sufficiently large (orders of mm) beam can hit a beam-pipe or other beamline elements
and will cause high rates in halo counters and/or beam loss monitors (BLMs) that will initiate the FSD.
The slow beam motion can be studied through MYA archiver, by correlating for example BPM positions with rates on Halo counters.

The fast beam motions (orders of microseconds or milliseconds) cannot be handled or studied with BPMs due to limited bandwidth of BMP readout amplifiers. The halo counters rates that are stored in MYA are also updating slow (mostly updated every 2 sec in HPS run period). In order to study beam motion at the microsecond level a system based on Struck SIS 3800 scaler was developed that allowed to buffer counter rates with $15~\mu$s dwell time. 
  
 \section{Connections and characteristics of Struck scalers}
 \indent
 
The system consisted of a hardware IOC with one Struck scaler (SIS 3800). A buffered readout of the scaler was organized in EPICS that allowed to write the buffer to
 a root file. Each buffer has 60000 elements, each element corresponds to the integrated rate in that channel with $15~ \mu$s dwell time.
 When the buffer is filled it takes about $1.1\; s$ to write them into a file. During that time scaler is blocked, doesn't count.
 Along with the $60\;K$ long buffer it also writes the time-stamp with a nano-second precision for each buffer that corresponds
 to the local time of the 1-st element of the buffer. Schematically the readout is presented in Fig.\ref{fig:Struck_Readout}.
  %%%%%%%%%%%%%%%%%%%%%%%%%%%%%%%%%%%%%%%%%%%%%% F I G U R E %%%%%%%%%%%%%%%%%%%%%%%%%%%%%%%%%%%%%%%%%%
 \begin{figure}[!htb]
  \centering
  \grinp[width=0.95\tw]{img/Struck_Readout.pdf}
  \caption{Readout of Struck scaler}
  \label{fig:Struck_Readout}
 \end{figure}
%%%%%%%%%%%%%%%%%%%%%%%%%%%%%%%%%%%%%%%%%%%%%% F I G U R E %%%%%%%%%%%%%%%%%%%%%%%%%%%%%%%%%%%%%%%%%%
  
% In the output Root File, 5 channels of Struck SIS 3800 are written. The correspondence of scaler channels and signals are shown in Table \ref{tb:scaler_signal_correspondance}.
%%%%%%%%%%%%%%%%%%%%%%%%%%%%%%%%%%%%%%%%%%%%%% T A B L E %%%%%%%%%%%%%%%%%%%%%%%%%%%%%%%%%%%%%%%%%%%% 
 \begin{table}[!htb]
 \centering
  \begin{tabular}{|c|c|}
   \hline
   Channel 0 & $25\; MHz$  clock \\ \hline
   Channel 1 & "HPS-SC" + ECal crystal \\ \hline
   Channel 2 & "Tagger-top" + "Upstr.-Right" + ECal crystal \\ \hline
   Channel 3 & "HPS-R" + ECal crystal \\ \hline
   Channel 4 &  ECal crystal \\ \hline
  \end{tabular}
  \caption{ Correspondence between scaler channels and signal sources.}
  \label{tb:scaler_signal_correspondance}
 \end{table}
%%%%%%%%%%%%%%%%%%%%%%%%%%%%%%%%%%%%%%%%%%%%%% T A B L E %%%%%%%%%%%%%%%%%%%%%%%%%%%%%%%%%%%%%%%%%%%% 
%%%%%%%%%%%%%%%%%%%%%%%%%%%%%%%%%%%%%%%%%%%%%% F I G U R E %%%%%%%%%%%%%%%%%%%%%%%%%%%%%%%%%%%%%%%%%%
\begin{figure}[!htb]
\centering
\grinp[width=0.9\tw]{img/Struck_Connections.pdf}
\caption{Schematic picture of connections: shown locations of all counters that are connected
to scaler.}
\label{fig:schematics_Connections}
\end{figure}
%%%%%%%%%%%%%%%%%%%%%%%%%%%%%%%%%%%%%%%%%%%%%% F I G U R E %%%%%%%%%%%%%%%%%%%%%%%%%%%%%%%%%%%%%%%%%%

Only the first 5 inputs of the scaler were used to readout a $25$ MHz pulser, 
HPS calorimeter modules, and a set of beam halo counters, see Table 
\ref{tb:scaler_signal_correspondance}. In Fig.\ref{fig:schematics_Connections} schematic view of the
locations of counters is shown.
  %%%%%%%%%%%%%%%%%%%%%%%%%%%%%%%%%%%%%%%%%%%%% F I G U R E %%%%%%%%%%%%%%%%%%%%%%%%%%%%%%%%%%%%%%%%%%
 \begin{figure}[!htb]
  \centering
  \grinp[width=0.75\tw]{img/25MHz_Clock.png}
  \caption{The 3 consequent buffers of $25 \; MHz$ pulser. In $15\;\mu s$ interval it corresponds to 375 counts. The gap between readings
  is the dead time, when scaler writes out the buffered data.}
  \label{fig:Clock_Channel}
 \end{figure}
%%%%%%%%%%%%%%%%%%%%%%%%%%%%%%%%%%%%%%%%%%%%% F I G U R E %%%%%%%%%%%%%%%%%%%%%%%%%%%%%%%%%%%%%%%%%%
 The channel-0 of the scaler was connected to the $25$ MHz pulser to be able to check whether the scaler reads and writes
 data properly.  In Fig.\ref{fig:Clock_Channel} the distribution of counts in channel-0 as a function of time. The 
 $25\; MHz$ in $15\;\mu s$ time interval corresponds to 375 counts.
 The gap between readings is the dead time, when scaler writes out the data into the file.
It should be noted that the time-stamps from buffer-to-buffer are not correct, the actual dead time is about $1.1\;s$, while
 in the Fig.\ref{fig:Clock_Channel} it appears as $\approx 150\; ms$.
 
\section{Testing the system with beam in the tagger dump} 
 \indent
 
 The first test of the system with the beam was done on March 14 and 15. For that tests beam was directed to the Hall-B tagger dump. At that time upstream BPMs were not calibrated yet.  
 Since the dwell time is quite small ($15\;\mu s$), one need to have a sufficient rates in the counters to be able to observe rate change over the small time interval. Therefore only upstream halo counters that are directly on the electron beamline can be used (scaler channel \#2). To get count rate high enough to be visible in the Struck readout the 1 mm thick iron wire on the "tagger harp" that is parallel to the $X$ axis was moved close to the beam to create about $3 \;MHz$ rates in the halo counters ("Upstr-R" and "Tgr-T" together).
      %%%%%%%%%%%%%%%%%%%%%%%%%%%%%%%%%%%%%%%%%%%%% F I G U R E %%%%%%%%%%%%%%%%%%%%%%%%%%%%%%%%%%%%%%%%%%
 \begin{figure}[!htb]
\centering
 \grinp[width=0.95\tw]{img/Upstream_Rate_Pos_Correlation.pdf}
 \caption{A figure showing periodic variation of rates on Upstr-R halo counter, when harp thick horizontal
 wire is touching the beam from the top. It is directly related to the
 beam Y position change at 2C24A.}
 \label{fig:upstr_rate_Variation.}
\end{figure}
 %%%%%%%%%%%%%%%%%%%%%%%%%%%%%%%%%%%%%%%%%%%%% F I G U R E %%%%%%%%%%%%%%%%%%%%%%%%%%%%%%%%%%%%%%%%%%
 
 In Fig.\ref{fig:upstr_rate_Variation.} about $4$ minute snapshot of various beam parameters during the tests from MYA archiver is shown. In the figure brown line is the beam current as measured on 2C24 nA-BPM, the green line is the Y-position of the bam on 2C24 and the red line rate on "Upstr-R" halo counter.
 As one can see the beam current in that period is quite stable, while the beam vertical position oscillates with a period of about
 $30$ s, and the rate in upstream halo counter oscillate with it. When beam moves up, and consequently touches
 the wire, rates quickly rise up to few $MHz$.
%%%%%%%%%%%%%%%%%%%%%%%%%%%%%%%%%%%%%%%%%%%%% F I G U R E %%%%%%%%%%%%%%%%%%%%%%%%%%%%%%%%%%%%%%%%%%
 \begin{figure}
  \centering
  \grinp[width=0.95\tw]{img/Upstream_rates1.png}
  \caption{An example of one buffer with 60K readings from scaller channel 2.}
  \label{fig:upstr_counts1}
 \end{figure}
%%%%%%%%%%%%%%%%%%%%%%%%%%%%%%%%%%%%%%%%%%%%% F I G U R E %%%%%%%%%%%%%%%%%%%%%%%%%%%%%%%%%%%%%%%%%% 
 
The fast beam motion studies using the Struck scaler system was done with above beam conditions.
 In Fig.\ref{fig:upstr_counts1} an example of one buffer with 60K readings from scaler channel \#2 is shown.
 As one can see rates on scaler oscillate a lot. Rates vary from 0 up to 140, where 
 140 counts in $15~\mu$s correspond to about $9\;MHz$. Taking into account that $9\;MHz$ is the sum of two counters
"Upstr-R" and "Tgr-t" and that these two counters count roughly equal, the maximum rate in each counter would correspond
to about $4.5\;MHz$. The rate of oscillations isn't appear to be constant, but it is mostly within $60\;Hz$ to $100\;Hz$.
So taking into account above mentioned one can conclude that, there
exist two beam motion components, one is fast, within $60\;Hz$ to $100\;Hz$, another
is slow with about 30 sec period. According to BPMs beam motion at 2C24 is about $500\; \mu m$. (One should note, 
however, that this doesn't mean that the beam motion will be the same order at the HPS detector,
since the optics is such that beam has the largest spatial size at 2C24,
whereas it is around $40\; \mu m$ in the 2H02A.)

\section{Beam motion studies at HPS}
\indent

The next set of beam motion studies with Struck scalers were performed on April 27. Similar to the upstream case the $1$ mm thick wire of the 2H02A 
harp 2H02 was placed close to the beam to create sufficient rates to on Struck scalers (in Ecal crystals and the HPS halo counters).
In Fig.\ref{fig:2H02_rates_pos_correl1} beam vertical position at 2H02 BPM (red) and the rate in HPS\_SC (green) are shown.
%%%%%%%%%%%%%%%%%%%%%%%%%%%%%%%%%%%%%%%%%%%%% F I G U R E %%%%%%%%%%%%%%%%%%%%%%%%%%%%%%%%%%%%%%%%%%
 \begin{figure}[!htb]
  \centering
  \grinp[width=0.95\tw]{img/BPM_Halo_2H02_1.pdf}
  \caption{Stripchart representing Rates on HPS\_SC (green) and 2H02 BPM\_Y (red) when 2H02A $1\;mm$ vertical wire is touching the beam from the top}
  \label{fig:2H02_rates_pos_correl1}
 \end{figure}
 %%%%%%%%%%%%%%%%%%%%%%%%%%%%%%%%%%%%%%%%%%%%% F I G U R E %%%%%%%%%%%%%%%%%%%%%%%%%%%%%%%%%%%%%%%%%%
This shows that the BPM noise level is around $50\;\mu m$. Looking carefully to the picture, one can notice 
that on top of BPM position fluctuations (the noise), there is also a small (w.r.t. BPM noise) position variations
that are correlated with rates on HPS\_SC.
%%%%%%%%%%%%%%%%%%%%%%%%%%%%%%%%%%%%%%%%%%%%% F I G U R E %%%%%%%%%%%%%%%%%%%%%%%%%%%%%%%%%%%%%%%%%%
% \begin{figure}[!htb]
%  \centering
%  \grinp[width=0.95\tw]{img/BPM_pos_width_curr_Correl.pdf}
%  \caption{Stripchart representing beam current (blue), beam position at 2H02A (red) and rates on
%  HPS\_SC (green) }
%  \label{fig:Width_curr_dep}
% \end{figure}
 %%%%%%%%%%%%%%%%%%%%%%%%%%%%%%%%%%%%%%%%%%%%% F I G U R E %%%%%%%%%%%%%%%%%%%%%%%%%%%%%%%%%%%%%%%%%%

 %%%%%%%%%%%%%%%%%%%%%%%%%%%%%%%%%%%%%%%%%%%%% F I G U R E %%%%%%%%%%%%%%%%%%%%%%%%%%%%%%%%%%%%%%%%%%
 \begin{figure}[!htb]
  \centering
  \grinp[width=0.95\tw]{img/Struck_Oscilations_Zoom.png}
  \caption{Counts as a function of time for Struck scaler channel 1 (HPS\_SC) in $100\;ms$ time
  interval. Each point represents counts in a $15\;\mu s$ time interval.}
  \label{fig:Struck_Oscil_Zoom1}
 \end{figure}
 %%%%%%%%%%%%%%%%%%%%%%%%%%%%%%%%%%%%%%%%%%%%% F I G U R E %%%%%%%%%%%%%%%%%%%%%%%%%%%%%%%%%%%%%%%%%%
 Looking rates with Struck scaler, we see a similar behavior as in the upstream case.
 In Fig.\ref{fig:Struck_Oscil_Zoom1} rate of channel \#1(HPS\_SC) as a function of time is shown for $100$ ms time interval.
 We see there is a clear oscillations of rates. 
 Oscillations are not occurring by a constant period, but it is mainly between $50$ to $100\;Hz$.
 From Fig.\ref{fig:2H02_rates_pos_correl1} we learnt that at 2H02 BPM beam motion is less than 50 $\mu m$ beam and the rate in HPS\_SC changes
 by $\approx 6$ times. In Fig.\ref{fig:Struck_Oscil_Zoom1} rate oscillations are the same order,
 about $5-6$ times between min and max values, therefore beam fast motion also should be within $50~ \mu$m.
  %%%%%%%%%%%%%%%%%%%%%%%%%%%%%%%%%%%%%%%%%%%%% F I G U R E %%%%%%%%%%%%%%%%%%%%%%%%%%%%%%%%%%%%%%%%%%
 \begin{figure}[!htb]
  \centering
  \grinp[width=0.95\tw]{img/Struck_trip2.png}
  \caption{Counts on HPS\_SC from Struck scaler as a function of time. No increase of rates is observed
  during the trip.}
  \label{fig:Trip_quiet1}
 \end{figure}
 %%%%%%%%%%%%%%%%%%%%%%%%%%%%%%%%%%%%%%%%%%%%% F I G U R E %%%%%%%%%%%%%%%%%%%%%%%%%%%%%%%%%%%%%%%%%%
 So behavior of the beam at upstream and at the HPS look similar. Both have slow and
 fast components, however the motion amplitude is much smaller, by factor of $10$, at the HPS (less than $50~ \mu$m).
 
Using the Struck data we also studied beam motion during the beam trips to see whether there is
 an increase of rates just before the trip. During the owl shift on April 27 we didn't observe any increase of
 rates before the trip. One example of the scaler rate when trip happened is shown in Fig.\ref{fig:Trip_quiet1}. As one can see that
 rates just go down to 0 within few $\mu s$. 
 
 \subsection{Some exception}
 \indent
 
 %%%%%%%%%%%%%%%%%%%%%%%%%%%%%%%%%%%%%%%%%%%%% F I G U R E %%%%%%%%%%%%%%%%%%%%%%%%%%%%%%%%%%%%%%%%%%
\begin{figure}[!htb]
\centering
 \subfloat[]{\label{fig:BPM_Jump_ex1}\grinp[width=0.95\tw]{img/BPM_Jump.pdf}} \\
 \subfloat[]{\label{fig:BPM_Jump_ex2}\grinp[width=0.95\tw]{img/BPM_Jump2.pdf}} \\
 \caption{Strip chart showing a significant beam movement for a few seconds. a) rates on HPS\_SC
 (green), 2H02.Y (pink) and rates on Downstream-bot (blue).
 b) Beam X and Y positions at 2C24 (green and yellow respectively), and at 2H02 (blue and pink respectively).}
 \label{fig_BPM_Jump1}
\end{figure}
%%%%%%%%%%%%%%%%%%%%%%%%%%%%%%%%%%%%%%%%%%%%% F I G U R E %%%%%%%%%%%%%%%%%%%%%%%%%%%%%%%%%%%%%%%%%%
 There were few instances when significant beam motion of the order
 of few hundreds of microns has been observed. One example is depicted in Fig.\ref{fig_BPM_Jump1}, where one can
 notice correlation between BPM position readings and the rates on halo counters, i.e. 
 HPS\_SC is saturated while downstream counter shows a clear peak during the beam movement.
 While at HPS beam movement is $\approx 150\; \mu m$ in vertical direction and $\approx 500\; \mu m$ in horizontal,
 on 2C24 BPM the beam position change in X is on oder few mm's. The fact that there is a big difference in beam motion amplitude between upstream of the orbit lock system
 (upstream of the 2H00 girder) indication that the movement is coming from the accelerator and the Hall-B orbit locks are able to compensate it.
 
\subsection{Studies with the target "IN"}
\indent

We took some parasitic data too with Struck scaler, i.e. during production running when target was on the beam.
Of course there will be a high rates present in the downstream halo counters that is coming from the target, but if beam moves 
and hits some thick/dense material then one should expect to see much higher rates on the halo counters.
%%%%%%%%%%%%%%%%%%%%%%%%%%%%%%%%%%%%%%%%%%%%% F I G U R E %%%%%%%%%%%%%%%%%%%%%%%%%%%%%%%%%%%%%%%%%%
\begin{figure}[!htb]
 \centering
 \subfloat[]{\label{fig:trip_quiet_target}\grinp[width=0.45\tw]{img/quiet_Target.png}}
 \subfloat[]{\label{fig:trip_quiet_semitarget}\grinp[width=0.45\tw]{img/Semiquiet_with_target.png}} \\
 \subfloat[]{\label{fig:trip_nonquiet_target_short}\grinp[width=0.45\tw]{img/Non_Queit_Target_quick.png}}
 \subfloat[]{\label{fig:trip_nonquiet_target_long}\grinp[width=0.45\tw]{img/Non_queit_Target.png}}
 \caption{Different scenarios of beam trips, when target was "IN". a) Trip is quiet, b) rates slightly increased
 before the trip. c) in $\approx 100\;\mu s$ rates increased by $\approx 10$ times before the trip,
 d) rates increased by $\approx 10$ times before the trip, then after $\approx 600 \;\mu s$ beam is tripped.}
 \label{fig:trips_with_target1}
\end{figure}
%%%%%%%%%%%%%%%%%%%%%%%%%%%%%%%%%%%%%%%%%%%%% F I G U R E %%%%%%%%%%%%%%%%%%%%%%%%%%%%%%%%%%%%%%%%%%
In Fig.\ref{fig:trips_with_target1} different scenarios of beam trips with the presence of the
target are shown. In Fig.\ref{fig:trips_with_target1}a beam trip is smooth, no rate increase before beam was completely off indicating that the beam didn't move enough to hit anything.  In Fig.\ref{fig:trips_with_target1}b counts were increased twice than normal by about $150~ \mu$s then went to 0.
This is likely due to the beam-tail interaction with some material, otherwise as wee saw before
rates would increase more than 10 times. In Fig.\ref{fig:trips_with_target1}c we see more than 10 times increase of rates again in a short
($\approx 150~\mu$s) time interval. The same as in previous case, beam had somewhat stronger interaction
 with some material before being completely off. The scenario in Fig.\ref{fig:trips_with_target1}d is quite rear in sense that
 the time period with high rates is about $600 \; \mu s$ whereas it is usually less than $100-200~\mu$s. 
 During almost all trips the trip type also has been recorded. We didn't observe any correlation between the trip
 type and the behavior of halo counters, e.g. for the RF trip sometimes beam goes off quietly without making
 high rates on the halo counters, sometimes rates may increase, jump by an order of magnitude
 before the trip. Same thing is with the BLM trips or other types of trips.
 We should stress that the increase of rates doesn't necessarily mean that the beam has interacted
 with the HPS collimator or with something around the HPS. We already saw from MYA and Struck data several instances where the beam has moved away upstream causing the rate increase every where. 
 
 
 \section{Testing of the FSD system}
 \indent
 
 It is clear from the above discussed that even large beam motions are not avoidable. 
 To keep SVT from being exposed by the beam, signals from all HPS halo counters (HPS\_L(R, SC, t)) have been summed up
 and sent to CEBAF Fast Shut Down (FSD) system. The working principal of the FSD board is quite simple, during some time window called integration time,
 it counts number of pulses on the input, and if it is higher than a certain threshold, it sends a signal to the injector to
 shut down the beam. The integration time and the threshold level can be set by the user. 
 Of course smaller the integration time better it is. But, in order to run with small integration times one should have an input to the system that
 will have sufficient counts in order to avoid frequent trips due to statistical fluctuations. 
The FSD card that was used has two limitations, integration time cannot be shorter than $1$ ms and the rates during the integration time can not be less than $1$ kHz.
 
 To test the FSD system we ran the 2H02A harp wire through the beam and recorded the rates with Struck scaler. Since the integration time is in order of milliseconds,  we increased the dwell time of Struck to be $300~\mu$s, which allowed to have larger number of counts in each time bin. In Fig.\ref{fig:Rates_FSD_Masked} rates of halo counters are shown when harp wire crossed the beam, the FSD was masked. In the figure the constant background 
 around 70 counts at the beginning and at the end are due to beam- target interactions, while starting at $t-t_{0}\approx 80 \;s$ 
 beam is hitting the wire and consequently creating higher rates in the counters. So, one should expect that if the FSD system was not masked this rate increase should have tripped the beam.  
 %%%%%%%%%%%%%%%%%%%%%%%%%%%%%%%%%%%%%%%%%%%%% F I G U R E %%%%%%%%%%%%%%%%%%%%%%%%%%%%%%%%%%%%%%%%%%
 \begin{figure}[!htb]
  \centering
  \grinp[width=0.95\tw]{img/FSD_Masked_Scan.png}
  \caption{Rates on HPS\_SC as a function of time when FSD was masked.}
  \label{fig:Rates_FSD_Masked}
 \end{figure}
 %%%%%%%%%%%%%%%%%%%%%%%%%%%%%%%%%%%%%%%%%%%%% F I G U R E %%%%%%%%%%%%%%%%%%%%%%%%%%%%%%%%%%%%%%%%%%
 %To determine what value to put for the threshold we run $1\;mm$ harp wire to cross the beam, while FSD was masked.
 
 First we tested the $10$ ms integration time. The threshold that we put is $3000$ counts in $10$ ms. This corresponds to $300\; KHz$ input rate, which translates
 into $90$ counts in $300\;\mu s$ time interval on the Struck.
 %%%%%%%%%%%%%%%%%%%%%%%%%%%%%%%%%%%%%%%%%%%%% F I G U R E %%%%%%%%%%%%%%%%%%%%%%%%%%%%%%%%%%%%%%%%%%
 \begin{figure}[!htb]
  \centering
  \subfloat[]{\label{fig:FSD_10ms}\grinp[width=0.95\tw]{img/FSD_10ms_30nA.pdf}}\\
  \subfloat[]{\label{fig:FSD_5ms}\grinp[width=0.95\tw]{img/FSD_Test.png}}
  \caption{HPS\_SC rates as a function of time. In top FSD integration time is $10\; ms$ and
  the the beam current is $30\;nA$. In the bottom graph FSD integration time is $5\; ms$ and
  the the beam current is $50\;nA$.}
  \label{fig:Rates_FSD_Test}
 \end{figure}
 %%%%%%%%%%%%%%%%%%%%%%%%%%%%%%%%%%%%%%%%%%%%% F I G U R E %%%%%%%%%%%%%%%%%%%%%%%%%%%%%%%%%%%%%%%%%%
 In Fig.\ref{fig:Rates_FSD_Test} two trips recorded with two integration times $10\; ms$ (top) and
 $5\;ms$ (bottom) are shown. Red line is the translated threshold that would be if integration time is
 equal to the dwell time of the Struck scaler ($300\;\mu s$). One can see that the time
 between threshold crossing and beam shutdown is consistent with the integration time.
 We repeated this test several times with both $5\;ms$ and $10\;ms$ integration times,  
 and observed consistent results. 
  \section{To summarize}
  \begin{itemize}
  
%  \item collimator scan with the beam suggests that collimator $3\;mm$ gap is about $3.45\;mm$. It will
%  more clear when collimator will be removed and gap size measured precisely.

 \item Two types of beam motions were observed, fast (roughly $100\; Hz$) and slow ($\sim 0.03$ Hz) that BPMs can detect. The latter one was handled by the feedback system (orbit locks).
\item  With engaged orbit locks the vertical beam motion at HPS is mainly within $50\;\mu m$.
  
 \item Orbit locks keep the beam quite stable, however sometimes it happens that during the trip, 
  or even without the trip beam moves by more than $100\;\mu m$ at HPS.

 \item The exact reason for the beam motion during the trip is not clear.
 Analyzes of different trips didn't find any correlations between the trip types and 
 the behavior of halo counters.

 \item The FSD system was tested, and it seems to work properly.

 \end{itemize}
 
%  \newpage
%  \clearpage
%  
%   \section{following will be removed, No need to read}
%   Downstream BPMs were calibrated and orbit locks were engaged on Apr 19 only,
% after the restoration of the machine from the power outage, that caused failure of CHL2.
% Next set of beam movement studies with Struck scalers began on Apr 27 owl.
% In a similar way as in upstream part, at HPS side also to determine
% the beam position, the 2H02A harp vertical thick wire ($1\;mm$ in diameter) was passed through the beam.
%  %%%%%%%%%%%%%%%%%%%%%%%%%%%%%%%%%%%%%%%%%%%%% F I G U R E %%%%%%%%%%%%%%%%%%%%%%%%%%%%%%%%%%%%%%%%%%
% \begin{figure}[!htb]
%  \centering
%  \grinp[width=0.95\tw]{img/FCup_harp_halo_correlation1.png}
%  \caption{Scan of the beam vertical profile with a harp $1\;mm$ wire. By blue shown harp motor position,
%  by green shown rates on HPS\_SC counter, by red shown rates on Downstream\_bottom counter
%  and the dashed orange line represents FCup current.}
% \label{fig:2H02A_thick_wire_scan}
% \end{figure}
%  %%%%%%%%%%%%%%%%%%%%%%%%%%%%%%%%%%%%%%%%%%%%% F I G U R E %%%%%%%%%%%%%%%%%%%%%%%%%%%%%%%%%%%%%%%%%%
%  In Fig.\ref{fig:2H02A_thick_wire_scan} shown time dependence of the Faraday Cup current (dashed orange), 
%  rates on HPS\_SC halo counter (green), rates on Downstream\_bottom counter (red) and the harp\_2H02A 
%  motor position (blue), during the scan.
%  One can notice an apparent correlation between these variables. When the harp wire is far from the beam
%  FCup current is stable (around $30\; nA$), both halo counters count quite small. As soon wire gets closer
%  to the beam FCup current start to decrease, both halo counters start to increase. Reaching $\approx 600\;KHz$ HPS\_SC
%  saturates causing counts to decrease. We don't see saturation on Downstream\_bottom, because it is upstream w.r.t. the 
%  harp\_2H02A, and consequently will have much lower rates (The peak value is $\approx 45\;Hz$).
%  The fact that FCup current decrease we think is because the $1\; mm$ wire plays a role of a small "beam blocker", and after the 
%  wire about $30\%$ of beam electrons have scattered at enough large angles, or have such low energy that chicane bends
%  them from the nominal trajectory causing them to miss FCup. For not being too busy plot the beam current measured at 2C24
%  is not plotted, but it is quite stable, confirming additionally that the FCup current change is not because of the actual
%  current decrease, but is because of the harp wire and the beam interaction.
%  %%%%%%%%%%%%%%%%%%%%%%%%%%%%%%%%%%%%%%%%%%%%% F I G U R E %%%%%%%%%%%%%%%%%%%%%%%%%%%%%%%%%%%%%%%%%%
%  \begin{figure}[!htb]
%   \centering
%   \grinp[width=0.95\tw]{img/FCup_Halo_corr_With_Fix_Harpwire.pdf}
%   \caption{Time dependence of HPS\_SC (green), FCup scaler (dashed orange), and beam X (dashed blue) and Y (Pink) in a 3 min
%   time interval.}
%   \label{fig:FCup_Halo_pos_time_dep1}
%  \end{figure}
%  %%%%%%%%%%%%%%%%%%%%%%%%%%%%%%%%%%%%%%%%%%%%% F I G U R E %%%%%%%%%%%%%%%%%%%%%%%%%%%%%%%%%%%%%%%%%%
%  After the scan, beam blocker was moved in front of the FCup, then by our request MCC delivered us $100\;nA$ beam, 
%  and harp vertical $1\;mm$ wire was positioned right at the place (around $10.55\;cm$) where rates on HPS\_SC start to increase.
%  Now when BPMs are calibrated and orbit locks are engaged, no periodic $30\;s$ rate oscillation was observed, 
%  as was observed in upstream part without engaged orbit locks. Actually 
%  during that shift few times the harp wire position was adjusted by $\approx 50\;\mu m$ to make sure that rates on HPS\_SC
%  counter are about $2-3\; MHz$, however this doesn't have a periodic character. 
%  In Fig.\ref{fig:FCup_Halo_pos_time_dep1} shown time dependence of HPS\_SC (green), 
%  FCup scaler (dashed orange), beam X (dashed blue) and Y (Pink) in a 3 min time interval. Note here instead of plotting FCup
%  measured current value, the FCup scaler value is plotted. This is because with beam blocker
%  the FCup current is small (around $0.9\; nA$), and small current changes are below MYA dead band (it is $0.01$ for this channel).
%  The FCup scaler value is almost linearly proportional to the FCup current, and therefore for correlation purposes 
%  it can be used instead of FCup current.
%  One can note that here also (when the beam blocker is under the beam), 
%  there is a correlation between FCup current and rates on Halo counters, however the correlation
%  now is opposite, i.e. increase of count rates correspond to the increase of FCup current.
%  The explanation for that we think is the following: when fraction of the beam (or entire beam) is interacting
%  with the harp wire, then after the wire the beam is not mono-energetic and parallel anymore, in addition to that
%  the chicane makes that dispersion even bigger, causing some part of the beam electrons escape the blocker and enter into FCUP.
%  %%%%%%%%%%%%%%%%%%%%%%%%%%%%%%%%%%%%%%%%%%%%% F I G U R E %%%%%%%%%%%%%%%%%%%%%%%%%%%%%%%%%%%%%%%%%%
%  \begin{figure}[!htb]
%   \centering
%   \grinp[width=0.95\tw]{img/FCup_Blocker_diagram.pdf}
%   \caption{A qualitative description of how the FCup current increase with the presence of the beam blocker,
%   when the beam interacts with a harp wire. Left represents the case when beam is not interacting with any matarial
%   before the beam blocker. Righ figure: beam has interacted with a harp wire.}
%   \label{fig:FCup_blocker_diagram}
%  \end{figure}
%  %%%%%%%%%%%%%%%%%%%%%%%%%%%%%%%%%%%%%%%%%%%%% F I G U R E %%%%%%%%%%%%%%%%%%%%%%%%%%%%%%%%%%%%%%%%%%
% It is qualitatively illustrated in Fig.\ref{fig:FCup_blocker_diagram}.
% Left drawing represents intact beam case, where almost all the beam hit the blocker, and in the
% right drawing beam has interacted with the harp before, and some of electrons will now miss the beam blocker and enter
% FCUP.
% In Fig.\ref{fig:FCup_Halo_pos_time_dep1} it is hard to see a correlation between beam
% vertical position and the rates on HPS\_SC mainly because the scale of vertical position is relatively large, and
% noise of BPM readings is large w.r.t. average position change.
% 
%  \subsection{Beam motions using the collimator}
%  Beam motions were studied with HPS collimator also. The idea was to find the ceter,
%  of the collimator $3\;mm$ gap w.r.t. the beam, then 
%  place the collimator bottom and top edges close to the beam by about $200\; \mu m$,
%  and the take some data with Struck scalers, to see whether there is an increase of rates
%  on beam trips or in general.
%   %%%%%%%%%%%%%%%%%%%%%%%%%%%%%%%%%%%%%%%%%%%%% F I G U R E %%%%%%%%%%%%%%%%%%%%%%%%%%%%%%%%%%%%%%%%%%
%   \begin{figure}[!htb]
%    \centering
%    \grinp[width=0.95\tw]{img/Colim_SCan.pdf}
%    \caption{collimator scan with the beam, to determine the center of the collimator $3\;mm$
%    gap w.r.t. the beam. In the figure shown motor position (blue), counts on HPS\_SC (gold),
%    and counts on HPS\_t (brown).}
%    \label{fig:Colim_Scan1}
%   \end{figure}
%   %%%%%%%%%%%%%%%%%%%%%%%%%%%%%%%%%%%%%%%%%%%%% F I G U R E %%%%%%%%%%%%%%%%%%%%%%%%%%%%%%%%%%%%%%%%%%
%  In Fig.\ref{fig:Colim_Scan1} shown the correlation between colimator motor position (blue) and 
%  rates on HPS\_SC (gold) and HPS\_t (brown). One can notice as cloimator top edge gets closer
%  to the beam, rates on halo counters gradually increase, and on the last step (motor is a $5.096^{"}$), rates increase
%  more than 10 times, that already means that the top edge of the colimator is interacting with
%  some fraction of the beam. In a similar way in the figure shown the determination of the motor
%  position where the bottom edge of the colimator starts to interact with the beam, which is $4.960^{"}$. 
%  The clomiator  center is defined as 
%  \begin{equation}
%   c = \frac{(5.096 + 4.960)}{2} = 5.028
%  \end{equation}
% One can calculate colimator gap width as $d = (5.096 - 4.960)\cdot 25.4 = 3.454\;mm$.
% During Struck studies we assumed $3\;mm$ gap and moved colimtor such that bottom and top 
% edges of the colimator be $\approx 200\;\mu m$ away from the beam, but since now it turns out that
% colimator gap is about $450 \mu m$ larger, the actual distance of the edges of the collimator from
% would be $200 + 450/2. = 425\;\mu m$. With each collimator position about $30$ minutes of Struck
% data was taken. During the whole 1 hour running. 6 beam trips were observed. Only in one case
% we saw increase of counts.
% %%%%%%%%%%%%%%%%%%%%%%%%%%%%%%%%%%%%%%%%%%%%% F I G U R E %%%%%%%%%%%%%%%%%%%%%%%%%%%%%%%%%%%%%%%%%%
% \begin{figure}[!htb]
%  \centering
%  \subfloat[]{\label{fig_colim_HPS_trip}\grinp[width=0.95\tw]{img/Colim_trip_HPS_SC.png}} \\
%  \subfloat[]{\label{fig_colim_upstream_trip}\grinp[width=0.95\tw]{img/Colim_trip_upstream.png}}
%  \caption{Couns on Upstream (top) and HPS\_SC (bottom) counters during the beam trip.}
%  \label{fig:Colim_trip1}
% \end{figure}
% %%%%%%%%%%%%%%%%%%%%%%%%%%%%%%%%%%%%%%%%%%%%% F I G U R E %%%%%%%%%%%%%%%%%%%%%%%%%%%%%%%%%%%%%%%%%%
% That case is represented in Fig.\ref{fig:Colim_trip1}. The graph on top represents HPS\_SC counter,
% and the graph on bottom represents "Upstream\_R + tagger\_t" (Reminder 15 counts is equivalent to $1\;MHz$).
% Since the peak present not only on HPS\_SC but also on upstream counters, then this means that
% the high rate is not only because of the interaction of the beam and the collimator but 
% the beam has interacted with some material before reaching the colimator. 
% 
% During that running we also observed similar peak on HPS\_SC, but in that case it wasn't related to the beam trip.
% Rates just peaked for a short period of time then decreased.
% %%%%%%%%%%%%%%%%%%%%%%%%%%%%%%%%%%%%%%%%%%%%% F I G U R E %%%%%%%%%%%%%%%%%%%%%%%%%%%%%%%%%%%%%%%%%%
% \begin{figure}
%  \centering
%  \grinp[width=0.95\tw]{img/Colim_Beam_move.png}
%  \caption{Counts on }
% \end{figure}
% %%%%%%%%%%%%%%%%%%%%%%%%%%%%%%%%%%%%%%%%%%%%% F I G U R E %%%%%%%%%%%%%%%%%%%%%%%%%%%%%%%%%%%%%%%%%%
%  There was no related peak observed at upstream part. That probably means that the peak is a result of the beam 
%  and the collimator interaction. In all 2 cases the characteristic time of the peak is less than $150\;\mu s$.
% 
% 
%   The BPM noise level is inversely proportional to the beam current. In the future runnings the nominal 
%  production current should be $\approx 200\;nA$, consequently
%  the spatial resolution of BPMs will be better, that will allow to estimate beam motions more precisely.

 \end{document}
 