The proposed $e\mu^+\mu^-$ final state will contain events from the $J/\psi$ electroproduction. measure the cross section of $J/\psi$ photoproduction 
on the proton near threshold ($E_{\gamma, {\rm threshold}} = 8.21$ GeV).

$J/\psi$ production near threshold is a rich and complex physics 
topic in its own right and presently the subject of intense theoretical 
research. The measurement of the $J/\psi$
photoproduction cross section is of great interest for the purpose of a
precise yield extraction. The projected results would, however, represent
a dramatic improvement over the world data on $J/\psi$ production near
threshold, and would thus impact the on-going theoretical discussions.
In this section we briefly describe the current understanding of 
$J/\psi$ production near threshold, the projected results, and the
role of the $J/\psi$ measurement in the present DDVCS experiment.


\subsection{$J/\psi$ production near threshold}
\label{subsec:JPsi}

The production of heavy quarkonia and their interaction with hadronic
matter are key questions of QCD, which are being studied through 
production experiments at different energies and various theoretical 
approaches; see Ref.~\cite{Brambilla:2010cs} for a recent review. 
Because of the small spatial size of heavy quarkonia on the hadronic 
scale, $r_{Q\bar Q} \ll 1~{\rm fm}$, one can use QCD operator methods
to describe their interactions with hadrons and external probes in
controlled approximation. Heavy quarkonium production probes the local 
color (gluon) fields in the nucleon, and can reveal properties such as 
their response to momentum transfer, their spatial distribution,
and their correlation with valence quarks. The dynamics that produces
the relevant gluon fields in the nucleon changes considerably
between high energies and the near-threshold region, creating a fascinating
landscape that calls for detailed experimental study. At high energies
($W > 10 \, {\rm GeV}$) exclusive $J/\psi$ photo-- and electroproduction 
probes the nucleon's gluon GPD at small momentum fractions 
$x \sim M_{J/\psi}^2/W^2 \ll 1$ and can be used to infer the transverse 
spatial distribution of small--$x$ gluons in the nucleon; 
see Ref.~\cite{frankfurt:2005mc} for a review; such experiments were 
performed at HERA \cite{aktas:2005xu,chekanov:2004mw} and FNAL 
\cite{binkley:1981kv}, and a detailed program of ``gluon imaging'' 
along these lines is planned with a future Electron Ion 
Collider (EIC) \cite{Boer:2011fh}. In exclusive $J/\psi$ production 
near threshold, the minimum invariant momentum transfer to the nucleon 
becomes large: $|t_{\rm min}| = 2.23$ GeV$^2$ at threshold, 
and $|t_{\rm min}| = 1.3-0.4$ GeV$^2$ in the $E_{\gamma} = 8.5-11$ 
GeV range. The process is therefore analogous to elastic $eN$ scattering 
at large $|t|$, only that the ``probe'' couples to the gluon field 
in the target. Exclusive $J/\psi$ production near threshold thus 
measures the nucleon form factor of a gluonic operator and can provide
unique information on the non-perturbative gluon fields in the nucleon.

The precise identification of the gluonic operators associated with
$J/\psi$ production near threshold and the modeling of their nucleon
form factors are the subject of intense theoretical research, the
status and perspectives of which were summarized at a recent topical
workshop \cite{temple}. Several approaches are presently being discussed. 
One scenario assumes that even near threshold the $J/\psi$ is produced 
through two--gluon exchange with a GPD--like coupling to 
the nucleon, but now in the special kinematics of large 
$|t| \sim |t_{\rm min}|$ and large ``skewness'' 
$\xi \sim 0.5$ \cite{frankfurt:2002ka}. A more likely possibility
is that the production process near threshold effectively reduces to 
a local gluonic operator, implying simple kinematic scaling 
relations~\cite{weiss:temple}. Another scenario uses the hard scattering 
mechanism for high-$t$ elastic form factors and assumes that the 
production process happens in the leading 3-quark 
Fock component of the nucleon, with rescattering through hard
gluon exchange~\cite{brodsky:2000zc}. $J/psi$ production near threshold 
is also being studied in the non--relativistic QCD (NRQCD) scheme,
which attempts a systematic parametric expansion in the heavy quark 
velocity~\cite{Butenschoen:2009zy,Butenschoen:2011yh}; first results 
for JLab 12 GeV kinematics were reported 
in Ref.~\cite{Butenschoen:temple}.

It is clear that progress with unraveling the mechanism of $J/\psi$ 
production near threshold depends crucially on experimental input.
Because of the small cross sections exclusive $J/\psi$ production 
near threshold was never measured with the precision necessary to 
discriminate between the proposed dynamical scenarios, let alone to 
extract quantitative information on the relevant operators probing 
the color fields in the nucleon. The existing data from the SLAC
and Cornell experiments~\cite{gittelman:1975ix,camerini:1975cy}
(see Fig.~\ref{fig:jpsixs}) provide some rough information on 
the energy dependence of the exclusive photoproduction cross section
and the $t$--slope near threshold. The present CLAS12 experiment
represents a unique opportunity to explore the unmeasured 
near--threshold region from $E_{\gamma} \approx 8.5 \, \textrm{GeV}$
to $11 \, \textrm{GeV}$. The projected data would dramatically
extend and improve our knowledge of the $J/\psi$ photoproduction
cross section and $t$--dependence near threshold (see Sect. \ref{sec:jprate}),
and directly impact on the on--going theoretical studies of the
reaction mechanism.


\subsection{Impact on the DDVCS measurement}
\label{sec:jpsi_ddvcs}


Since the final states for DDVCS and J/$\psi$ are identical, the detector efficiency and resolution for exclusive $J/\psi$
production is very similar to that of DDVCS events in the proposed range of lepton invariant mass. The narrow peak of the $J/\psi$ will make it easy to identify the reaction, and more suitable for a reliable yield extraction than the DDVCS-BH continuum. The $J/\psi$ electroproduction
reaction can thus serve as an important benchmark, allowing us to better
understand the systematic uncertainties.
The $\phi(1020)$ could in principle also be used in a similar way at the
lower end of the invariant mass range.
A measurement of the $J/\psi$ cross section in parallel with DDVCS will thus
be very beneficial for the understanding the DDVCS data, and help addressing
the two main sources of systematic uncertainty, i.e. acceptance
and the muon identification.
%This is discussed further in Sect.~\ref{sec:systematics}.



