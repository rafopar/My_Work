
To understand whether we will have good enough acceptance and reasonable rates, we performed Monte Carlo simulations 
of the $ep\rarr e^{\prime}\mu^{-}\mu^{+}p$ reaction. The incoming beam energy is assumed to be $11 \; GeV$.
%%%%%%%%%%%%%%%%%%%%%%%%%%%%%%%%%%%%%%% F I G U R E %%%%%%%%%%%%%%%%%%%%%%%%%%%%%%%%%%%%%%%%%%%%%%%%%
\begin{figure}[!htb]
 \centering
 \begin{subfigure}{0.48 \tw}
  \grinp[width=0.99\tw]{img/Qp2_Q2_gen_general2.pdf}
  \caption{}
  \label{fig:Qp2_Q2_gen_general2}
 \end{subfigure}
 \begin{subfigure}{0.48 \tw}
  \grinp[width=0.99\tw]{img/Qp2_Q2_rec_general2.pdf}
  \caption{}
  \label{fig:Qp2_Q2_rec_general2}
 \end{subfigure}
\begin{subfigure}{0.48 \tw}
  \grinp[width=0.99\tw]{img/Q2_xB_smear2.pdf}
  \caption{}
  \label{fig:Q2_xB_smear2}
 \end{subfigure}
 \begin{subfigure}{0.48 \tw}
  \grinp[width=0.99\tw]{img/tM_smear4.pdf}
  \caption{}
  \label{fig:tM_smear4}
 \end{subfigure}
 \caption{Top row: Generated (left) and reconstructed (right) distributions of "$Q^{\prime 2}$ vs $Q^{2}$".
 Bottom row: reconstructed "$Q^{2}$ vs $x_{B}$" (left) and $-t$ (right) when  $Q^{2} \in(2.5 - 3)\;GeV^{2}$ and $x_{B}\in(0.11 - 0.12)$.}
 \label{fig:kinematics_general}

\end{figure}
%%%%%%%%%%%%%%%%%%%%%%%%%%%%%%%%%%%%%%% F I G U R E %%%%%%%%%%%%%%%%%%%%%%%%%%%%%%%%%%%%%%%%%%%%%%%%%
Simulations were limited in the $Q^{2}\in(0.8-7.2)\;GeV^{2}$ and $Q^{\prime 2}\in(1.8-4.8)\;GeV^{2}$ region.
Generated DDVCS events were passed through the CLAS12 FastMC Acceptance package (described in a previous section).
In Fig.\ref{fig:kinematics_general} shown distributions of some kinematic variables: Generated (top left) and reconstructed (top right) 
"$Q^{\prime 2}$ vs $Q^{2}$" distributions, "$Q^{2}$ vs $x_{B}$" is the bottom left, and the bottom right represent $-t$ distribution 
when $Q^{2}$ and $x_{B}$ are in the following region $Q^{2} \in(2. - 3)\;GeV^{2}$ and $x_{B}\in(0.11 - 0.2)$.

In this LOI we will present estimated rates and statistical uncertainties of beam spin asymmetries as a function of
$\Phi_{L}$ by fixing kinematic bin $Q^{2} \in(2. - 3.)\;GeV^{2}$, $-t\in (0.1 - 0.4)\; GeV^{2}$, $x_{B}\in(0.11 - 0.2)$ (See 
bottom row of Fig.\ref{fig:kinematics_general}), and 
varying $Q^{\prime 2}$ ($Q^{\prime 2} = {2.,\; 2.8,\;  3.6} \;GeV^{2}$). Limits are also shown in 

%%%%%%%%%%%%%%%%%%%%%%%%%%%%%%%%%% MC generator description %%%%%%%%%%%%%%%%%%%%%%%%%%%%%%%%%%%%%%%%
Each generated event contains 7 fold cross section, which is used for estimation of rates.
\begin{equation}
\frac{\dstl d\sigma}{\dstl dQ^{2} dt dQ^{\prime 2} dx_{B} d\Phi_{L} d\Phi_{cm} d\Theta_{cm}} 
\end{equation}
Where angles $\Phi_{L}$, $\Phi_{CM}$ and $\Theta_{CM}$ are shown on Fig. \ref{fig:DDVCS_Angles}, and their definition is described in the caption
of the figure.
%%%%%%%%%%%%%%%%%%%%%%%%%%%%%%%%%%%%%%% F I G U R E %%%%%%%%%%%%%%%%%%%%%%%%%%%%%%%%%%%%%%%%%%%%%%%%%
\begin{figure}[!htb]
 \centering
 \begin{subfigure}{0.48 \tw}
  \grinp[width=0.99\tw]{Angles_DDVCS1.pdf}
  \caption{}
  \label{fig:DDVCS_Angles1}
 \end{subfigure}
 \begin{subfigure}{0.48 \tw}
  \grinp[width=0.99\tw]{Angles_DDVCS2.pdf}
  \caption{}
  \label{fig:DDVCS_Angles2}
 \end{subfigure}
\caption{Representation of DDVCS angles. $\Phi_{L}$ is the angle between beam scattering and hadronic planes (left figure), $\Phi_{CM}$ 
is the angle between decay lepton and hadronic planes (left figure) and $\Theta_{CM}$ (right figure) is the angle of the negative
decay lepton w.r.t. scattered proton momentum, in the frame where timelike photon is at rest.}
\label{fig:DDVCS_Angles}
\end{figure}
%%%%%%%%%%%%%%%%%%%%%%%%%%%%%%%%%%%%%%% F I G U R E %%%%%%%%%%%%%%%%%%%%%%%%%%%%%%%%%%%%%%%%%%%%%%%%%

The cross section represent the sum of three cross sections DDVCS, BH and their interference term.
%Calculation of cross section is not fast and it is very unpractical to calculate cross section for each generated event,
%instead the cross section is calculated in a discrete values of $Q^{2}$, $Q^{\prime 2}$, $t$, $\Phi_{L}$, $\Phi_{CM}$ and $\Theta_{CM}$, then
%cross section is extrapolated using the two (in each variable) closest values of already defined cross section.
Kinematic distribution of final state particles, when all of them ($e^{-},\;\mu^{-}$ and $\mu^{+}$) are detected, are shown in 
Fig.\ref{fig:particle_kinematic_distr}, where
\begin{figure}[!htb]
 \centering
  \begin{subfigure}{0.48 \tw}
  \grinp[width=0.99\tw]{img/th_P_mum_rec4.pdf}
  \caption{}
  \label{fig:th_P_mum_rec4}
 \end{subfigure}
 \begin{subfigure}{0.48 \tw}
  \grinp[width=0.99\tw]{img/th_P_mup_rec4.pdf}
  \caption{}
  \label{fig:th_P_mup_rec4}
 \end{subfigure}
 \begin{subfigure}{0.48 \tw}
  \grinp[width=0.99\tw]{img/th_P_em_rec2.pdf}
  \caption{}
  \label{fig:th_P_em_rec2}
 \end{subfigure}
 \begin{subfigure}{0.48 \tw}
  \grinp[width=0.99\tw]{img/MM_smear1.pdf}
  \caption{}
  \label{fig:MM_smear1}
 \end{subfigure}
 \caption{ $\theta$ vs $P$ distributions for detected $\mu^{-}$ (a), $\mu^{+}$ (b) and $e-$ (c).
(d) is the missing mass of detected $e^{-}\mu^{-}\mu^{+}$ system.}
\label{fig:particle_kinematic_distr}
\end{figure}
a, b, and c represent "$\theta$ vs $P$" distributions for $\mu^{-}$, $\mu^{+}$ and $e^{-}$ respectively, and d is the
missing mass of detected $e^{-}\mu^{-}\mu^{+}$ system.
As one can see because of different kinematics of $J/\Psi$ (narrow and high $M(\mu^{-}\mu^{+})$ mass region) and DDVCS (wide and lower $M(\mu^{-}\mu^{+})$),
kinematic distributions "$\theta$ vs $P$" are not quite similar,
however missing mass resolution is still as good as in $J/\Psi$
case (see Figs. \ref{fig:jp_mukine} and \ref{fig:jp_mres}), and will allow ensure exclusivity of the reaction.

For the aforementioned kinematic bin, the acceptance and expected rates in three bins of $Q^{\prime 2}$ are shown in Fig.\ref{fig:Acc_and_rates_Qp2}
As one can see the acceptance is not bad and it is around $2.5\%$.
\begin{figure}[!htb]
 \centering
   \begin{subfigure}{0.48 \tw}
  \grinp[width=0.99\tw]{img/Acc_Qp2.pdf}
  \caption{}
  \label{fig:Acc_Qp2}
 \end{subfigure}
 \begin{subfigure}{0.48 \tw}
  \grinp[width=0.99\tw]{img/h_Qp2_rates1.pdf}
  \caption{}
  \label{fig:h_Qp2_rates1}
 \end{subfigure}
\caption{Acceptance and expected rates for different bins of $Q^{\prime 2}$ (left)}
\label{fig:Acc_and_rates_Qp2}
\end{figure}
Later to estimate statistical uncertainties on the Beam Spin asymmetries, each $Q^{\prime 2}$ bin is divided into 12 bins on $\Phi_{L}$, and for each bin 
counts and statistical error-bars on asymmetry as a function of $\Phi$
 are shown in \ref{fig:counts_and_errorbars0}, \ref{fig:counts_and_errorbar1} and \ref{fig:counts_and_errorbars2}.
 These estimations were performed assuming $10^{37}cm^{-2}s^{-}$ luminosity and 100 days of running.
%%%%%%%%%%%%%%%%%%%%%%%%%%%%%%%%%%%%%%% F I G U R E %%%%%%%%%%%%%%%%%%%%%%%%%%%%%%%%%%%%%%%%%%%%%%%%%
\begin{figure}[!htb]
 \centering
   \begin{subfigure}{0.48 \tw}
  \grinp[width=0.99\tw]{img/PHI_DVCS_rates_0.pdf}
  \caption{}
  \label{fig:PHI_DVCS_rates_0}
 \end{subfigure}
 \begin{subfigure}{0.48 \tw}
  \grinp[width=0.99\tw]{img/Asym_errorbars_0.pdf}
  \caption{}
  \label{fig:Asym_errorbars_0}
 \end{subfigure}\\
\caption{Count rates (left) and statistic error-bars for (right) for the $Q^{\prime2}(1.8-2.4); GeV^{2}$ bin}
\label{fig:counts_and_errorbars0}
\end{figure}
%%%%%%%%%%%%%%%%%%%%%%%%%%%%%%%%%%%%%%% F I G U R E %%%%%%%%%%%%%%%%%%%%%%%%%%%%%%%%%%%%%%%%%%%%%%%%%
\begin{figure}[!htb]
 \centering
   \begin{subfigure}{0.48 \tw}
  \grinp[width=0.99\tw]{img/PHI_DVCS_rates_1.pdf}
  \caption{}
  \label{fig:PHI_DVCS_rates_1}
 \end{subfigure}
 \begin{subfigure}{0.48 \tw}
  \grinp[width=0.99\tw]{img/Asym_errorbars_1.pdf}
  \caption{}
  \label{fig:Asym_errorbars_1}
 \end{subfigure}\\
\caption{Count rates (left) and statistic error-bars (right) for the $Q^{\prime2}(2.4 - 3.2); GeV^{2}$ bin}
\label{fig:counts_and_errorbar1}
\end{figure}
%%%%%%%%%%%%%%%%%%%%%%%%%%%%%%%%%%%%%%% F I G U R E %%%%%%%%%%%%%%%%%%%%%%%%%%%%%%%%%%%%%%%%%%%%%%%%%
\begin{figure}[!htb]
 \centering
   \begin{subfigure}{0.48 \tw}
  \grinp[width=0.99\tw]{img/PHI_DVCS_rates_2.pdf}
  \caption{}
  \label{fig:PHI_DVCS_rates_2}
 \end{subfigure}
 \begin{subfigure}{0.48 \tw}
  \grinp[width=0.99\tw]{img/Asym_errorbars_2.pdf}
  \caption{}
  \label{fig:Asym_errorbars_2}
 \end{subfigure}\\
\caption{Count rates (left) and statistic error-bars (right) for the $Q^{\prime2}(3.2 - 4); GeV^{2}$ bin}6
\label{fig:counts_and_errorbars2}
\end{figure}
%%%%%%%%%%%%%%%%%%%%%%%%%%%%%%%%%%%%%%% F I G U R E %%%%%%%%%%%%%%%%%%%%%%%%%%%%%%%%%%%%%%%%%%%%%%%%%
% \begin{figure}[!htb]
%  \centering
%    \begin{subfigure}{0.48 \tw}
%   \grinp[width=0.99\tw]{img/PHI_DVCS_rates_3.pdf}
%   \caption{}
%   \label{fig:PHI_DVCS_rates_3}
%  \end{subfigure}
%  \begin{subfigure}{0.48 \tw}
%   \grinp[width=0.99\tw]{img/Asym_errorbars_3.pdf}
%   \caption{}
%   \label{fig:Asym_errorbars_3}
%  \end{subfigure}\\
% \caption{Count rates (left) and statistic error-bars (right) for the $Q^{\prime2}(4. - 4.8); GeV^{2}$ bin}
% \label{fig:counts_and_errorbars3}
% \end{figure}
%%%%%%%%%%%%%%%%%%%%%%%%%%%%%%%%%%%%%%% F I G U R E %%%%%%%%%%%%%%%%%%%%%%%%%%%%%%%%%%%%%%%%%%%%%%%%%