%\documentclass[letterpaper,12pt]{article}
\documentclass[superscriptaddress,preprintnumbers,amsmath,amssymb,12pt]{revtex4}

%\usepackage{graphics}
\usepackage[dvipdf]{graphicx}
\usepackage{subfig}  % For subfloats
\usepackage{color}
\usepackage{multirow}
\usepackage{epsfig}
\usepackage{wrapfig}
\usepackage{rotating}

\def \rarr {\rightarrow}
\def \grinp {\includegraphics}
\def \tw {\textwidth}
\def\dfrac#1#2{\displaystyle{{#1}\over{#2}}}
\def \dstl {\displaystyle}
\definecolor{GREEN}{rgb}{0.,0.8,0}
\definecolor{RED}{rgb}{1,0,0}
\definecolor{ORANGE}{rgb}{1,0.5,0}

\newcommand{\JLAB}{Thomas Jefferson National Accelerator Facility, Newport News, Virginia 23606}
\newcommand{\ORSAY}{Thomas Jefferson National Accelerator Facility, Newport News, Virginia 23606}

\begin{document}
\rightline{LOI to PAC43}
\vspace{1.cm}

\title{Di-muon electroproduction with 11GeV electron beam and the modified CLAS12 detector}

\author{M. Guidal, E. Voutier, M. Boer} 
%\affiliation{\JLAB}
\author{N. Baltzell, V. Burkert, L. Elouadrhiri, F-X. Girod, S. Stepanyan, M. Ungaro} 
%\affiliation{\JLAB}
\author{R. Paremuzyan  ...}

\date{\today}


\begin{abstract}
The current program for studying Generalized Parton Distributions (GPDs) using the JLAB 12 GeV machine consists of measuring spin (beam/target) observables and cross sections in Deeply Virtual Compton Scattering (DVCS) and Deeply Virtual Meson Production (DVMP), and the angular asymmetries in the Timelike Compton Scattering (TCS). In these reactions, observables contain integrals of GPDs over the quark momentum fraction $x$ (real part of the amplitude) or at specific kinematical point ($x=\xi$) (imaginary part of the amplitude). In contrast, the Double Deeply Virtual Compton Scattering (DDVCS), where both the incoming and outgoing photons have high virtuality, allows to map out GPDs in wide range of $x\ne \xi$. The proposed experimental program with longitudinally polarized 11 GeV electron beam and the CLAS12 detector in Hall-B aims to study GPDs for quarks in the valence region using DDVCS and investigate the gluonic structure of the nucleon through J/$\Psi$ electroproduction using the reaction $e ~p\to e^\prime ~p^\prime \gamma^\star(J/\Psi)\to e^\prime ~p^\prime \mu^+\mu^-$. 

%The CLAS12 Central Detector will not be used in this experiment. 

\end{abstract}

\maketitle


\end{document}