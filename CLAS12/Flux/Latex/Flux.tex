\documentclass[letterpaper,12pt]{article}
%documentclass[superscriptaddress,preprintnumbers,amsmath,amssymb,aps,11pt]{revtex4}
%\usepackage[]{authblk}
%\usepackage{graphics}
\usepackage[dvipdf]{graphicx}
%\usepackage{subfig}  % For subfloats
\usepackage{color}
%\usepackage[backend=bibtex]{biblatex}
\usepackage{url}
\usepackage{epsfig}
\usepackage{wrapfig}
\usepackage{rotating}
\usepackage{caption}
\usepackage{subcaption}

\oddsidemargin = -14mm
\topmargin = -2.9cm
\textwidth = 19cm
\textheight = 24cm

\def \rarr {\rightarrow}
\def \grinp {\includegraphics}
\def \tw {\textwidth}
\def\dfrac#1#2{\displaystyle{{#1}\over{#2}}}
\def \dstl {\displaystyle}
\definecolor{GREEN}{rgb}{0.,0.8,0}
\definecolor{RED}{rgb}{1,0,0}
\definecolor{ORANGE}{rgb}{1,0.5,0}


\begin{document}
 \section{ Photon Flux }
 In the electroproduction when the virtuality of the transferred photon is small $Q^{2} \sim 0$,  
 the electroproduction cross-section can be parametrized as a product of the photoproduction cross-section and the
 virtual photon flux.
 \begin{equation}
  \sigma_{e} = N_{\gamma}(E_{\gamma})\sigma_{\gamma}
 \end{equation}
 where $\sigma_{e}$ is the electroproduction cross-section, $\sigma_{\gamma}$ is
 the photoproduction cross-section, and the $N_{\gamma}(E_{\gamma})$ is a flux of virtual photon, that represents
 number of virtual photons in the unit range of photon energy. This is known as
 Equivalent Photon Approximation (EPA).
 
 In the electroproduction experiments, beam electrons will produce Bremsstrahlung photons in the target also,
 the flux of that photons is proportional to the target length.
 When the target length is not very small (w.r.t radiation  length of the target), then real photon flux
 can be significant and one needs to consider that also.
 
 The effective flux of photns therefore will be the sum of quasi-real (Virtual photon) and real
 (Bremsstrahlung) photon fluxes.
 
 The virtual photon flux is calculated using the following
 equation \cite{Phot_Flux1}
 \begin{equation}
  N(E_{\gamma}) = \frac{1}{E_{b}} \frac{\alpha}{\pi\cdot x}
  \cdot\left( (1 - x + \frac{x^{2}}{2})\cdot log(\frac{Q^{2}_{max}}{Q^{2}_{min}}) - (1 - x)\right)
 % (1/Eb)*\alpha/(PI*x)*( (1 - x + x*x/2)*log(\frac{Q2_{max}}{Q2_{min}}) - (1 - x));
 \end{equation}

where $E_{b}$ is the beam energy, $E_{\gamma}$ is the photon energy, $x = \dfrac{E_{\gamma}}{E_{b}}$, 
$Q^{2}_{min}$ is kinematically allowed minimum value of $Q^{2}_{min}$, and $Q^{2}_{max}$ is the
cutoff value on $Q^{2}$. 
 
 When electron passes a matter of the length $dx$, the number of Bremsstrahlung photons
 with a good approximation can be calculated using the following formula \cite{PDG} (section: passage of
 particles through the matter).
 \begin{equation}
  N(E_{\gamma}) = \frac{dx}{X_{0}}\frac{1}{E_{\gamma}}\cdot\left( \frac{4}{3} - \frac{4}{3}\frac{E_{\gamma}}{E_{b}} 
   + \frac{E_{\gamma}^{2}}{E_{b}^{2}}\right)
   \label{eq:Bremsstr}
 \end{equation}
Here $X_{0}$ is the radiation length.

In the Bremsstrahlung case, when the photon is produced at x distance from the beginning of the target,
then it will travel only $l-x$ distance in the target ($l$  is the target length). Taking into account this,
in the luminosity calculation one need to take the integral $\dstl \int_{0}^{l} N(E_{\gamma})\cdot(l - x) dx $.
After calculating the integral, one finds that the effective flux will be
 \begin{equation}
  N(E_{\gamma}) = \frac{l}{2\cdot X_{0}}\frac{1}{E_{\gamma}}\cdot\left( \frac{4}{3} - \frac{4}{3}\frac{E_{\gamma}}{E_{b}} 
   + \frac{E_{\gamma}^{2}}{E_{b}^{2}}\right)
   \label{eq:Bremsstr_eff}
 \end{equation}

\begin{figure}[!htb]
 \centering
 \grinp[width=0.75\tw]{img/Virt_Brem_Fluxes.pdf}
 \caption{ Virtual (red) and Real (blue) photon fluxes as a function of photon energy.}
 \label{fig:fluxes}
\end{figure}

 In. Fig.\ref{fig:fluxes} shown virtual (red) and real (blue) photon fluxes as a function of photon energy.
 Note that in the calculations target langth was taken $15\;cm$ and the $Q^{2}_{max} = 0.1\; GeV^{2}$.
 For the total integrated flux in the $E_{\gamma}\in(9. - 11)\; GeV$ range, we estimated
 $N =$ $0.00241179$ (virtual) $+\; 0.00153735$ (real) $ = 3.949\times 10^{-3}$ photons per electron.
 
 \begin{thebibliography}{45}
  \bibitem{Phot_Flux1} \url{ http://lss.fnal.gov/archive/other/lpc-94-35.pdf}
  \bibitem{PDG} Chin. Phys. C, 38, 090001 (2014)
 \end{thebibliography}

\end{document}
